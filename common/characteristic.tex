
{\actuality} В качестве предмета исследования и анализа в диссертации выступают объекты наукометрической информационно-аналитической системы, которые описываются наборами ключевых слов. Кроме того, объекты такой системы связаны между собой различными отношениями, например, для научный публикации это может быть список соавторов, для сотрудника - список проектов, в которых он работал, или список конференций, в которых он принимал участие. Публикации, персоналии, проекты и конференции в данном примере являются объектами и, следовательно, могут иметь собственные наборы ключевых слов. 

Побудительным мотивом и конечной целью исследований, результаты которых представлены в настоящей диссертации, является создание интеллектуального программного модуля, встраиваемого в наукометрическую информационную систему, и способного по имеющимся в системе данным и связям выявлять семантическую информацию и решать задачи информационного поиска и классификации. Отмечается также, что зачастую информационные системы не обладают большим объемом данных для анализа, поэтому важным аспектом в работе рассматриваемого модуля является способность эффективно работать в условиях недостаточности входной информации.

{\aim}  диссертационной работы является исследование и разработка математических моделей, алгоритмов и программных средств интеллектуального анализа наборов ключевых слов, характеризующих объекты в наукометрических интеллектуальных системах, с использованием методов из теории графов и дополнительной онтологической информации об объектах в системе. Такая деятельность соответствует областям исследования, отмеченным в пп. 1, 2, 5, 9 Паспорта специальности 05.13.17 – теоретические основы информатики.


{\novelty}
автором разработаны новые алгоритмы определения семантической близости для пары ключевых слов, а также для пары наборов ключевых слов, описывающих объекты интеллектуальной наукометрической системы. Созданы уникальные методы автоматической генерации обучающей выборки, а также методы автоматической проверки качества работы программных реализаций алгоритмов определения семантически похожих ключевых слов и алгоритмов выявления кластеров близких понятий, что является важным, поскольку тестирование программ данной области требовательно к наличию специалистов, способных точно определить степень близости для пары объектов или понятий. Разработаны алгоритмы построения иерархических классификаторов научных направлений в автоматическом режиме, использующие исключительно наборы ключевых слов. Важными особенностями  указанных алгоритмов являются: отсутствие необходимости больших объемов данных для обучения моделей с приемлемым уровнем качества; возможность использования разработанных моделей для произвольных интеллектуальных систем, использующих ключевые слова для описания сущностей; небольшие человеческие трудозатраты для выставления экспертных оценок. Проведена работа по уменьшению числа параметров системы, что делает разработанные модели легкими для настройки.

На основе исследованний о семантической похожести ключевых и имеющейся онтологической информации об объектах решены задачи поиска эксперта и кластеризации научных сотрудников.

Доказана вычислительная сложность разработанных алгоритмов.

{\influence} Рассматриваемый в работе программный комплекс для анализа, обработки и поиска объектов интеллектуальных информационных систем по ключевым словам представляет собой самостоятельный инновационный продукт. Он может использоваться не только в системе, рассматриваемой в данной диссертации, но и в любой информационно-аналитической системе, объекты которой описываются наборами ключевых слов. Кроме того, разработанные методики обработки связей между объектами могут быть перенесены на другие задачи анализа взаимосвязанных объектов. Рассматриваемый программный модуль определения семантической близости между словами  порождает словарь синонимов той области, на которой был обучен. Этот словарь может быть использован в самых разнообразных задачах информационного поиска и обработки естественного языка, и потенциально может привнести дополнительный полезный сигнал для моделей классификации, ранжирования и кластеризации текстовых или текстово-аннотированных объектов.

{\methods} В работе применяются методы анализа текстов на естественном языке, методы машинного обучения и программной инженерии. При изложении результатов диссертационной работы широко используется аппарат теории графов, а также математической логики и математической статистики.

{\probation} Результаты диссертации докладывались на всероссийской конференции с международным участием <<Знания–Онтологии–Теории (ЗОНТ-2016)>>, на международной конференции <<Ломоносовские чтения>> (2014, 2016, 2018),  на механико-математическом факультете МГУ имени М.В. Ломоносова на семинаре «Проблемы современных информационно-вычислительных систем» под руководством д.ф.-м.н., проф. В.А. Васенина (2013, 2015, 2017, 2018).

%{\contribution} Автор принимал активное участие \ldots

