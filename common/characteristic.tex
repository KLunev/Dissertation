
{\actuality} В качестве предмета исследования и анализа в диссертации выступают объекты наукометрической информационно-аналитической системы, которые описываются наборами ключевых слов. Кроме того, объекты такой системы связаны между собой различными отношениями, например, для \hl{научной} публикации это может быть список соавторов, для \hl{научного работника - список проектов, в выполнении которых он принимал участие}, или список конференций, в которых он принимал участие. Публикации, персоналии, \hl{научные проекты} и конференции в данном примере являются объектами и, следовательно, могут иметь собственные наборы ключевых слов. 

Побудительным мотивом и конечной целью исследований, результаты которых представлены в настоящей диссертации, является создание интеллектуального программного модуля, встраиваемого в наукометрическую \hl{информационно-аналитическую систему, способного по имеющимся в системе ключевым словам и определенным связям между ними выявлять семантическую информацию и с ее помощью решать задачи информационного поиска и классификации. Следует также отметить то обстоятельство, что зачастую информационные системы не обладают большим объемом данных для анализа, поэтому важным требованием к разрабатываемому модулю является его способность эффективно работать в условиях ограниченного объема входной информации.}

{\aim}  диссертационной работы является исследование и разработка математических моделей, алгоритмов и программных средств интеллектуального анализа наборов ключевых слов, характеризующих объекты в наукометрических интеллектуальных системах, с использованием методов из теории графов и дополнительной \hl{информации онтологического характера} об объектах в системе. Такая деятельность соответствует областям исследования, отмеченным в пп. 1, 2, 5, 9 Паспорта специальности 05.13.17 – теоретические основы информатики.

{\workrequirements}
\hl{Согласно стандарту \mbox{ГОСТ Р ИСО/МЭК 9126-93} к качеству разрабатываемой системы интеллектуального анализа объектов информационной системы предъявляются следующие требования:}
\begin{itemize}
    \item функциональность; 
    \item надежность;
    \item практичность;
    \item эффективность;
    \item сопровождаемость;
    \item мобильность.
\end{itemize}

\hl{Более подробно вышеизложенные пункты определены в приложении {\ref{AppendixRequirements}}. Описанные в данном приложении характеристики определяют отличительные стороны решаемой в настоящей диссертации задаче от известных работ, связанных с выделением семантической информации между объектами информационных систем. Существующие системы в большинстве своем опираются на обилие входных данных:}
\begin{itemize}
    \item \hl{текстовая информация: аннотации, заголовки, полные тексты документов;}
    \item \hl{общие объемы системы: значительное количество сущностей внутри системы и число связей между ними;}
\end{itemize}

\hl{В то же время разрабатываемый программный комплекс является более гибким решением для небольших систем. Сущности таких систем должны лишь обладать описывающим их набором ключевых слов, либо быть соединены внутренними связями с сущностями, которым набор ключевых слов ассоциирован. Кроме того, разработанные подходы позволяют получать узконаправленные семантические модели для конкретной области знаний. Ручной труд при внедрении таких систем сводится к минимуму.}

\hl{В разделе {\ref{related_work_concl}} содержится мотивация предъявленных требований к системе, разрабатываемой в данной работе. Кроме того, приводятся недостатски существующих методов решения подобных задач. Описываются проблемные места, которые не позволяют применять эти подходы к некоторому классу систем. В конечном итоге выделяется \textbf{специфика} разрабатываемого комплекса, отличающего его он аналогов. По имеющимся требованиям была создана методология решения поставленной задачи.}

{\methods} 
\hl{Для достижения поставленных целей и удовлетворения описанных выше требований были рассмотрены различные методы решения, их преимущества и недостатки. По окончании поиска была составлена методология исследования,  наиболее подходящая поставленным в настоящей диссертации задачам в рамках имеющихся особенностей и ограничений в наборов исходных данных. Подробная мотивация выбранной методологии описывается в {\ref{methodology}}}

%\hl{Первым важным решением было использование графовой схемы представления данных. При этом во внимание принимались следующие аргументы:}
%\begin{itemize}
%    \item \hl{в связи с тем, что информационная система может не располагать большими объемами данных, важно в полной мере использовать каждый объект и каждую связь в системе;}
%    \item \hl{для этого подходящим является графовое представление данных, поскольку по графам можно восстановить частичную недостаточность данных. Отсутствие реальной связи между объектами в силу незаполненности системы можно восполнить, проанализировав пути и отношения в графе;}
%    \item \hl{в следствие небольших объемов входных данных, графовое представление является эффективным способом хранения информации;}
%    \item \hl{между сущностями возможны разные типы отношений. Построение одного графа для нескольких таких типов может позволить эффективно пользоваться взаимодействием между этими отношениями;}
%    \item \hl{графовое представление открывает огромные возможности по подсчету различных характеристик для пары сущностей, оцениваемых на уровень похожести. Появляется возможность подсчета пути, количеств различных путей, мер центральностей, соседств, потоков, различных графовых индексов, кластерного анализа  и использования всего богатого инструментария теории графов. Все это может оказать положительный эффект на качество определения семантической близости.}
%\end{itemize}

%\hl{. То есть сначала решаются базовые задачи определения близости пары слов, затем процесс поднимается на более высокий уровень определения близости наборов, после чего происходит переход к поставленным в диссертации задачам. Мотивировка данного решения излагается далее.}


\hl{Была принята следующая методология решения:}
\begin{enumerate}
    \item \hl{данные системы представляются в виде множества графов, вершинами которых являются некоторые понятия (ключевые слова/наборы слов/сущности системы), а ребрами - отношения между ними. По построенным графам подсчитываются различные характеристики для пар вершин;}
    \item \hl{решается задача определения семантической близости пары ключевых слов. Для этого используются построенные графы, разработанные подходы и технологии машинного обучения. Решению этой задачи посвящена глава {\ref{chapt_word_similarity}};}
    \item \hl{решается задача определения семантической близости пары наборов ключевых слов. Разработанные модели используют различные графовые представления, подходы и модели, рассмотренные в предыдущих пунктах. Моделям, решающие указанную задачу описаны в главе {\ref{chapt_tuple_similarity}}.}
    \item \hl{используя функцию близости наборов ключевых слов и отношения между сущностями системы, решаются прикладные задачи определения семантической близости пары сущностей. Этой задаче посвящается глава {\ref{chapt_applications}}}
\end{enumerate}
В работе применяются методы анализа текстов на естественном языке, методы машинного обучения и программной инженерии. При изложении результатов диссертационной работы широко используется аппарат теории графов, а также математической логики и математической статистики.

{\novelty}
\hl{работы определяется тем, что автором разработаны} новые алгоритмы определения семантической близости для пары ключевых слов, а также для пары наборов ключевых слов, описывающих объекты интеллектуальной наукометрической системы. Созданы уникальные методы автоматической генерации обучающей выборки, а также методы автоматической проверки качества работы программных реализаций алгоритмов определения семантически похожих ключевых слов и алгоритмов выявления кластеров близких понятий\hl{. Последнее обстоятельство важно, поскольку тестирование программ в данной предметной области требовательно к наличию специалистов}, способных точно определить степень близости для пары объектов или понятий. Разработаны алгоритмы построения иерархических классификаторов научных направлений в автоматическом режиме, использующие исключительно наборы ключевых слов. Важными особенностями  указанных алгоритмов являются: отсутствие необходимости больших объемов данных для обучения моделей с приемлемым уровнем качества; возможность использования разработанных моделей для произвольных интеллектуальных систем, использующих ключевые слова для описания сущностей; небольшие человеческие трудозатраты для выставления экспертных оценок. Проведена работа по уменьшению числа параметров системы, что делает \hl{разработанные модели и программные средства эргономичными и легкими для настройки.}

На основе исследованний о семантической похожести ключевых и \hl{дополнительной информации онтологического характера об объектах, подлежащих анализу, решен ряд значимых задачи и востребованных на практике задач.}

Доказана вычислительная сложность разработанных алгоритмов, \hl{подтверждающая их адекватность (соответствие) требованиям, предъявляемым к разрабатываемому программному комплексу.}

{\influence} Рассматриваемый в работе программный комплекс для анализа, обработки и поиска объектов интеллектуальных информационных систем по ключевым словам представляет собой самостоятельный инновационный продукт. Он может использоваться не только в системе, рассматриваемой в данной диссертации, но и в любой информационно-аналитической системе, объекты которой описываются наборами ключевых слов. Кроме того, разработанные методики обработки связей между объектами могут быть перенесены на другие задачи анализа взаимосвязанных объектов. Рассматриваемый программный модуль определения семантической близости между словами  порождает словарь синонимов той области, на которой был обучен. Этот словарь может быть использован в самых разнообразных задачах информационного поиска и обработки естественного языка, и потенциально может привнести дополнительный полезный сигнал для моделей классификации, ранжирования и кластеризации текстовых или текстово-аннотированных объектов.


{\probation} Результаты диссертации докладывались на всероссийской конференции с международным участием <<Знания–Онтологии–Теории (ЗОНТ-2016)>>, на международной конференции <<Ломоносовские чтения>> (2014, 2016, 2018),  на механико-математическом факультете МГУ имени М.В. Ломоносова на семинаре «Проблемы современных информационно-вычислительных систем» под руководством д.ф.-м.н., проф. В.А. Васенина (2013, 2015, 2017, 2018).

%{\contribution} Автор принимал активное участие \ldots

