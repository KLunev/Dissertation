\chapter*{Заключение}						% Заголовок
\addcontentsline{toc}{chapter}{Заключение}	% Добавляем его в оглавление

%% Согласно ГОСТ Р 7.0.11-2011:
%% 5.3.3 В заключении диссертации излагают итоги выполненного исследования, рекомендации, перспективы дальнейшей разработки темы.
%% 9.2.3 В заключении автореферата диссертации излагают итоги данного исследования, рекомендации и перспективы дальнейшей разработки темы.
%% Поэтому имеет смысл сделать эту часть общей и загрузить из одного файла в автореферат и в диссертацию:

\hl{В настоящей диссертации представлено описание методологии реализованных подходов к анализу данных, характеризующих объекты в  информационно-аналитической наукометрической системе. Предложены модели, алгоритмы и реализующие их программные средства. В основе разработанных подходов лежат методы теории графов, анализа текстов на естественном языке, машинного обучения и программной инженерии.}

Отмечается, что предложенные автором подходы к построению моделей семантического анализа могут применяться не только для наукометрических систем, но также и для других систем, объекты которой описываются набором слов естественного языка. Важным преимуществом разработанных моделей является возможность их эффективного внедрения в системы, не обладающие большим объемом исходных данных. Другое их достоинство заключается в способности использовать произвольные связи между объектами системы для улучшения качества определения уровня семантической близости.

В ходе работ по подготовке настоящей диссертации получены следующие \textbf{основные результаты}.
\begin{itemize}
    \item Разработан ряд моделей вычисления уровня семантической близости между ключевыми словами интеллектуальной системы. Программные реализации моделей позволяют вычислять такую близость, автоматически учитывая специфику системы, в которую модели внедряются. Проведены многочисленные тестовые испытания, подтверждающие высокий уровень полученных результатов. Получены аналитические оценки, характеризующие вычислительную сложность программных реализаций этих моделей.
    \item Разработана модель и ее программная реализация для вычисления семантической близости между парой объектов. Для решения задачи были использованы дополнительные связи между сущностями системы. Модель протестирована на данных из ИАС <<ИСТИНА>>. Показано, что модель удовлетворяет предъявляемым к ней требованиям.
    \item На основе разработанных моделей решены важные, востребованные практикой задачи поиска экспертов в различных областях научных знаний, кластеризации ключевых слов, \hl{определения тематической направленности объекта информационно-аналитической системы}. Программные реализации решений опробированы на данных из ИАС <<ИСТИНА>>, включены в состав ее программного обеспечения, а также получили высокие оценки качества квалифицированных экспертов.
\end{itemize}

\textbf{Благодарности.} Автор выражает огромную благодарность своему научному руководителю доктору физико-математических наук, профессору Валерию Александровичу Васенину за внимание к работе на всех ее этапах, редакторскую правку диссертационной работы и опубликованных статей, а также за ценные наставления, терпение и понимание.

Автор благодарит кандидата физико-математических наук, доцента С.А.Афонина за активное учаситие в постановке задач, за плодотворное обсуждение результатов работы, а также за техническую помощь в проведении исследований и экспериментов.

%% Согласно ГОСТ Р 7.0.11-2011:
%% 5.3.3 В заключении диссертации излагают итоги выполненного исследования, рекомендации, перспективы дальнейшей разработки темы.
%% 9.2.3 В заключении автореферата диссертации излагают итоги данного исследования, рекомендации и перспективы дальнейшей разработки темы.
%\begin{enumerate}
%  \item Результат 1 \ldots
%  \item Результат 2 \ldots
%  \item Результат 3 \ldots
%  \item Результат 4 \ldots
%\end{enumerate}

