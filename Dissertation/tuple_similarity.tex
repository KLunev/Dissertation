\chapter{Определение смысловой близости пары наборов ключевых слов}
В данном разделе представлены алгоритмы выявления смысловой близости между двух наборами ключевых слов и, как следствие, уровень близости пары объектов интеллектуальной системы, с которыми были ассоциированы данные наборы ключевых слов.
Для этого используется словари синонимов, составление которых подробно описано в предыдущей главе.
В конце раздела представлены результаты тестовых испытаний программных реализаций алгоритмов, выводы о выполненной работе, а также предлагаются идеи для дальнейшего улучшения качества определения семантической близости наборов ключевых слов.

\section{Алгоритмы определения смысловой близости наборов ключевых слов}
\section{Алгоритмы определения смысловой близости коротких предложений}
\section{Методы кластеризации наборов ключевых слов}

\section{Решение задачи поиска экспертов} \label{expert_search_tuplesim}
\subsection{Определение близости наборов для решения задачи поиска экспертов}
В качестве меры близости пары наборов ключевых слов автором предлагается следующая формула:
$$ TupleSim_{expert}(X,Y) = \frac{\sum_{i=1}^{|X|}\sum_{j=1}^{|Y|}WordSim_{expert}(X_i, Y_j)}{|X \bigcup Y|}, $$

где $|\cdot|$ ­ количество слов в наборе, $X_i$, $Y_j$ ­ $i$­ый и $j$­ый теги наборов $X$, $Y$ соответственно. $WordSim_{expert}$ - мера близости, введенная в \ref{expert_search_wordsim}. Числитель этой формулы аккумулирует близость всех пар слов из разных наборов. Если положить $WordSim_{expert}(x, y) = \mathbbm{1}x=y$, то числитель будет равен числу общих тегов, что приведет к более простой модели вычисления близости по мере Жаккара. Без нормировки длинные пары наборов были бы сильнее похожи друг на друга, чем короткие пары.


\section{Тестовые испытания}
...
\section{Выводы}
...
