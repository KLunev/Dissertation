\input{Dissertation/appendixsetup}   % Предварительные настройки для правильного подключения Приложений
\chapter{Требования к качеству программной системы анализа ключевых слов } \label{AppendixRequirements}
Настоящее приложение содержит характеристики и показатели, определяющие требования, которые предъявляются к качеству разрабатываемого программного комплекса в соответствии со стандартом \mbox{ГОСТ Р ИСО/МЭК 9126-93}.
\section{Функциональность}
\hl{Системой должен быть поддержан следующий функционал:}
\begin{enumerate}
    \item  \hl{подготовка моделей определения семантической близости пар объектов системы по имеющимся данным:}
    \begin{enumerate}
        \item  \hl{модель семантической близости пары ключевых слов;}
        \item  \hl{модель семантической близости наборов ключевых слов;}
        \item  \hl{модель семантической близости пары сущностей системы.}
    \end{enumerate}
    \hl{Самым вычислительно сложным является построение модели близости пары ключевых слов. Каждая следующая модель обучается последовательно, поскольку в значительной степени опирается на предыдущую. Весь этап обучения является алгоритмически сложной задачей и включает в себя несколько этапов:}
        \begin{itemize}
            \item \hl{предподготовка данных;}
            \item \hl{построение необходимых графов из данных;}
            \item \hl{сбор обучающей выборки для моделей;}
            \item \hl{подсчет многочисленных графовых характеристик по графам для объектов обучающей выборки;}
            \item \hl{непосредственно обучение модели.}
        \end{itemize}
    \item \hl{быстрое обновление имеющихся и добавление новых данных в систему:}
    \begin{enumerate}
        \item \hl{добавление/модификация наборов ключевых слов;}
        \item \hl{добавление/модификация дополнительных графов, связывающих сущности системы различными отношениями.}
    \end{enumerate}
    \hl{При изменении данных системы возникает необходимость переобучения моделей близости для поддержания консистентного состояния между данными и моделями. Сложность данного пункта в том, что наивная переподготовка моделей после изменения данных может занимать продолжительное время. По этой причине возникает необходимость разработки сложных алгоритмов инкрементальной подготовки моделей, что сократит время их дообучения.}
    \item \hl{быстрая кластеризация ключевых слов системы и поиск необходимого кластера;}

    \hl{Данный процесс происходит после обучения модели близости пары ключевых слов.}
    \item \hl{быстрый поиск похожих объектов с помощью обученных моделей:}
    \begin{enumerate}
        \item \hl{поиск наиболее похожих ключевых слов к заданному;}
        \item \hl{поиск сущностей, релевантных заданному набору ключевых слов;}
        \item \hl{поиск набора ключевых слов, подходящих для заданной сущности.}
    \end{enumerate}
    \item \hl{реализация подмодулей, решающих практически значимые задачи информационного поиска в рамках аналитической системы:}
    \begin{enumerate}
        \item \hl{подмодуль поиска эксперта. Реализация функционала поиска сущностей информационной системы, релевантных поисковому запросу из ключевых слов;}
        \item \hl{подмодуль предложенных ключевых слов. Реализация функционала предложения пользователю новых ключевых слов по словам, введенным на данный момент или по имеющейся связанной информации.}
    \end{enumerate}
    \item \hl{сбор пользовательской информации в ходе взаимодействия с комплексом:}
    \begin{enumerate}
        \item \hl{подмодуль поиска эксперта. Логирование релевантных и нерелевантных по мнению пользователя результатов;}
        \item \hl{подмодуль предложенных ключевых слов. Логирование выбранных и невыбранных пользователем ключевых слов из числа предложенных.}
    \end{enumerate}
    \hl{Поиск должен выполняться как только пользователь ввел запрос и подтвердил его. Вычисление и показ результатов должны укладываться в несколько секунд. Сложность данного требования в том, что для каждого запроса необходимо подсчитать огромное число графовых характеристик и применить соответствующую предобученную графовую модель близости. В следствии этого данный пункт представляет собой сложную техническую задачу по оптимизации вычислений.}
    \item \hl{cоответствие принятым в индустрии соглашениям и стандартам;}
    \item \hl{высокий уровень доверия и защищенность ресурсов системы от деструктивных воздействий путем соблюдения положений (стандартов) информационной и функциональной безопасности, а также использования программного обеспечения с открытым исходным кодом;}
    \item \hl{система должна функционировать под управлением ОС с открытым исходным кодом.}
\end{enumerate}

\section{Надежность}
Следующие свойства должны быть удовлетворены:
\begin{enumerate}
    \item  \hl{качество обученных моделей должно валидироваться на отложенных выборках после каждого изменения моделей}
        \begin{enumerate}
            \item \hl{для каждой модели и соответствующих ей наборов тестов определяется необходимый уровень качества по выбранным метрикам и уровень производительности и величину ресурсозатратности;}
            \item \hl{для каждой модели выбирается отложенное множество объектов, на которых модель применяется. При обновлении данных, переобученные модели применяются к тому же множеству объектов и автоматически проверяется, что изменения в предсказаниях оказываются ниже определенного порога. Если это условие не выполняется, то эксперту по системе необходимо детально разбираться в причинах сильных отклонений в предсказаниях. Таким образом в системе реализуется регрессионное тестирование.}
        \end{enumerate}
    \item  \hl{cтабильная работа в условиях одновременного использования сотрудниками крупной организации;}
    \item  \hl{устойчивость к программным ошибкам и ошибкам интерфейса;}
\end{enumerate}
\section{Практичность}
В отношении разрабатываемого комплекса должно выполняться следующее:
\begin{enumerate}
    \item  \hl{комплекс должeн иметь простой интуитивный интерфейс для пользователя;}
    \item  \hl{комплекс должeн быть легко читаемой и понимаемой для разработчиков;}
    \item  \hl{комплекс должен включать средства обратной связи пользователя с разработчиками.}
\end{enumerate}
\section{Эффективность}
Программный комплекс должен быть эффективен в следующих показателях:
\begin{enumerate}
    \item  \hl{этап предподготовки комплекса:}
        \begin{enumerate}
            \item В течение одних суток:
            \begin{enumerate}
                \item \hl{пересчет аналитических моделей определения близости ключевых слов, включая подготовку всех необходимых данных;}
                \item \hl{пересбор тезауруса ключевых слов;}
            \end{enumerate}
            \item В течение нескольких часов:
            \begin{enumerate}
                \item \hl{обогащение наборов ключевых слов новой информацией;}
                \item \hl{пересчет аналитических моделей определения близости объектов информационной системы;}
                \item \hl{быстрое добавления новых отношений между сущностями системы;}
            \end{enumerate}
        \end{enumerate}
    \item  \hl{этап использования моделей:}
        \begin{enumerate}
            \item  \hl{быстрое построение выдачи по пользовательскому запросу;}
            \item  \hl{быстрое получение кластера ключевых слов содержащее данное;}
            \item  \hl{быстрая реализация поисковых подсказок при вводе запроса;}
        \end{enumerate}
\end{enumerate}
\section{Сопровождаемость}
Выдвигаются следующие требования к разрабатываемому комплексу по сопровождаемости:
\begin{enumerate}
    \item  \hl{весь комплекс архитектурно должен разбиваться на ряд отдельных модулей. Логика и параметры этих модулей системы должны быть инкапсулированы друг от друга;}
    \item  \hl{иметь возможность быстрого и эффективного способа расширения функционала комплекса;}
    \item  \hl{иметь возможность обновлять входные данные в автоматическом режиме;}
    \item  \hl{быть документированной;}
\end{enumerate}
\section{Мобильность}
Следующие свойства должны выполняться для разрабатываемого комплекса:
\begin{enumerate}
    \item \hl{возможность внедрения в различные информационно-аналитические системы произвольной направленности с допустимым уровнем качества моделей. Модели должны иметь возможность обучаться на данных новой системы;}
    \item \hl{возможность обучения специфических моделей семантической близости, автоматически подстраиваемых к предметной области системы, в которой разворачивается комплекс;}
    \item \hl{возможность обучения моделей семантической близости без имеющихся обучающих примеров;}
    \item \hl{возможность внедрения в систему с дефицитом данных о ключевых словах;}
    \item \hl{адаптируемость к добавлению новых сущностей и отношений между ними в системе;}
    \item \hl{развертываемость комплекса внутри новой системы не должна занимать много времени работы экспертов. Необходимо лишь наладить поставку данных в нужном формате и сконфигурать модули для наиболее эффективного решения задач конкретной системы;}
    \item \hl{устойчивость к пропускам и неточностям в данных.}
\end{enumerate}

Описанные выше требования задают специфику разрабатываемому программному комплексу. Главные особенности заключаются в следующем:
\begin{itemize}
    \item \hl{Комплекс может быть внедрен в систему, не обладающую достаточными объемами данных;}
    \item \hl{Комплекс поддерживает добавление произвольных отношений различной природы между сущностями}
\end{itemize}

\chapter{Самые абстрактные по смыслу слова для каждой меры
центральности} \label{AppendixA}
Для каждого алгоритма выписаны 50 самых абстрактных ключевых слов. Жирным шрифтом выделены слова, которые, по мнению авторов, не должны попадать в список самых абстрактных в рамках исследуемого корпуса слов, т.е. ошибочно определённые слова.
\begin{itemize}
    \item \textbf{Betweenness Centrality}: моделирование, модель, структура, оптимизация, математическая модель, математическое моделирование, управление, \textbf{мониторинг}, образование, прогнозирование, эксперимент, \textbf{прочность}, методы, методика, \textbf{самоорганизация}, история, \textbf{адаптация}, \textbf{здоровье}, \textbf{синтез}, анализ, \textbf{эффективность}, свойства, диагностика, \textbf{инновации}, \textbf{оценка}, технология, \textbf{устойчивость}, безопасность, личность, \textbf{надежность}, компьютерное моделирование, \textbf{взаимодействие}, динамика, качество, термодинамика, \textbf{плазма}, \textbf{наночастицы}, развитие, исследование, культура, \textbf{лазер}, теория, интеграция, модернизация, \textbf{деформация}, \textbf{метод конечных элементов}, \textbf{конкурентоспособность}, численное моделирование, \textbf{студенты}, алгоритм.
    \item \textbf{Closeness Centrality}: модель, моделирование, структура, \textbf{оптимизация}, управление, прогнозирование, методика, эксперимент, анализ, математическая модель, методы, математическое моделирование, \textbf{мониторинг}, \textbf{эффективность}, \textbf{надежность}, качество, технологии, \textbf{прочность}, расчет, \textbf{оценка}, планирование, \textbf{инновационная культура}, исследование, инновации, синтез, \textbf{устойчивость}, \textbf{взаимодействие}, образование, проектирование, безопасность, обучение, динамика, свойства, деформация, информационная система, \textbf{самоорганизация}, \textbf{инновационная деятельность}, вероятность, \textbf{профессионализм}, эксплуатация, \textbf{здоровье}, интеграция, инновационное развитие, кинетика, \textbf{температура}, \textbf{вуз}, \textbf{адаптация}, \textbf{работоспособность}, история, алгоритм.
    \item \textbf{Degree Centrality}: моделирование, математическая модель, математическое моделирование, \textbf{оптимизация}, модель, образование, управление, структура, мониторинг, \textbf{личность}, \textbf{прочность}, инновации, свойства, прогнозирование, \textbf{эффективность}, \textbf{синтез}, методика, культура, \textbf{метод конечных элементов}, безопасность, \textbf{оценка}, компьютерное моделирование, \textbf{наночастицы}, \textbf{развитие}, \textbf{адаптация}, эксперимент, \textbf{студенты}, \textbf{здоровье}, качество, история, \textbf{анизотропия}, \textbf{надежность}, технология, \textbf{компетентностный подход}, \textbf{инновационная деятельность}, численное моделирование, диагностика, модернизация, разрушение, \textbf{конкурентоспособность}, творчество, интеграция, \textbf{высшая школа}, \textbf{компетенции}, \textbf{самоорганизация}, \textbf{устойчивость}, динамика, \textbf{вуз}, \textbf{остаточные напряжения}, кинетика.
    \item \textbf{EigenVector Centrality}: образование, управление, модель, инновации, моделирование, \textbf{эффективность}, \textbf{инновационная деятельность}, наука, \textbf{личность}, методика, \textbf{оптимизация}, модернизация, технологии, прогнозирование, мониторинг, компетенции, государство, конкурентоспособность, структура, развитие, интеграция, математическая модель, качество, \textbf{оценка}, анализ, история, высшая школа, культура, взаимодействие, студенты, \textbf{надежность}, инновационное развитие, методы, \textbf{власть}, \textbf{бизнес}, \textbf{вуз}, стратегия, \textbf{компетенция}, эксперимент, \textbf{инновационная культура}, обучение, планирование, \textbf{бакалавриат}, общество, \textbf{компетентностный подход}, \textbf{здоровье}, \textbf{инновационный потенциал}, математическое моделирование, концепция, проект.
    \item \textbf{PageRank Centrality}: моделирование, математическая модель, математическое моделирование, \textbf{оптимизация}, модель, образование, \textbf{мониторинг}, структура, управление, \textbf{метод конечных элементов}, прогнозирование, \textbf{прочность}, \textbf{наночастицы}, компьютерное моделирование, \textbf{личность}, \textbf{эффективность}, инновации, развитие, диагностика, численное моделирование, методика, безопасность, \textbf{компетентностный подход}, культура, \textbf{синтез}, \textbf{адаптация}, свойства, \textbf{здоровье}, \textbf{оценка}, \textbf{устойчивость}, технология, \textbf{надежность}, \textbf{разрушение}, \textbf{наноструктуры}, \textbf{студенты}, интеграция, история, \textbf{роман}, динамика, \textbf{анизотропия}, \textbf{профессиональное образование}, кинетика, алгоритм, \textbf{плазма}, \textbf{вуз}, \textbf{конкурентоспособность}, качество, \textbf{качество образования}, \textbf{остаточные напряжения}, \textbf{дистанционное обучение}.
\end{itemize}

\chapter{Приложение Б. Найденные в коллекции документов тематические теги} \label{AppendixB}
Жирным шрифтом выделены те теги, которые определены верно.

эпр, медь, алтай, \textbf{аудит}, музей, поиск, \textbf{право}, доходы, охрана, смазка, стресс, тьютор, услуги, \textbf{физика}, катализ, матрица, порошок, контекст, покрытия, преграда, адсорбция, \textbf{биометрия}, коррекция, облучение, \textbf{семантика}, \textbf{кинематика}, \textbf{статистика}, предприятие, детали машин, станки с чпу, тестирование, фитопланктон, гидродинамика, дальний восток, самореализация, \textbf{конструирование}, диоксид циркония, жидкие кристаллы, пограничный слой, \textbf{факторный анализ}, \textbf{массовая культура}, преподаватель вуза, имитационная модель, управление знаниями, \textbf{нелинейные колебания}, \textbf{регрессионный анализ}, \textbf{электронное обучение}, ресурсное обеспечение, электроэнцефалограмма, \textbf{оптимальное управление}, \textbf{физическое моделирование}, образовательная программа, образовательные технологии, поддержка принятия решений, высокоскоростное соударение, \textbf{педагогическая деятельность}, международное сотрудничество, научно-образовательный центр, профессиональные компетенции, система менеджмента качества, экспериментальные исследования, \textbf{нелинейные динамические системы}, \textbf{финансово-хозяйственная деятельность}, федеральный государственный образовательный стандарт, nanoparticles.

Некоторые теги не определяют название дисциплины или направления, но по ним также можно понять тематику документа. Поэтому считается разумным отнести к правильно определенным тематическим тегам следующие:

охрана, покрытия, коррекция, облучение, детали машин, дальний восток, самореализация, управление знаниями, ресурсное обеспечение, образовательная программа, образовательные технологии, профессиональные компетенции, система менеджмента качества.

Далее представлены результаты работы программной реализации алгоритма на данных из Веб.

\textbf{trade}, \textbf{testing}, \textbf{principal component analysis}, mechanical properties, microstructure, heterogeneity, identification, globalization, \textbf{semantic web}, turkey, australia, sensors, information, oxidative stress, wireless sensor networks, tracking, \textbf{privacy}, \textbf{sustainable development}, \textbf{architecture}, feature extraction, obesity, apoptosis, conservation, \textbf{pattern recognition}, \textbf{risk assessment}, \textbf{kinetics}, poverty, india, depression, \textbf{cryptography}, climate, diagnosis, virtual reality, parameter estimation, gene expression, collaboration, \textbf{policy}, chaos, detection, finite element method, breast cancer, copper, \textbf{optimal control}, algorithms, mems, memory, decomposition, concrete, xml, usa, corrosion, taxonomy, \textbf{dynamic programming}, planning, volatility, aggregation, \textbf{spectroscopy}, russia, \textbf{dynamics}, density, mobility, dna, \textbf{cfd}, \textbf{sensitivity analysis}.

Аналогично случаю с чистыми данными, можно дополнить список следующими словами:

mechanical properties, microstructure, wireless sensor networks, virtual reality.
