\input{Dissertation/appendixsetup}   % Предварительные настройки для правильного подключения Приложений
\chapter{Самые абстрактные по смыслу слова для каждой меры
центральности} \label{AppendixA}
Для каждого алгоритма выписаны 50 самых абстрактных ключевых слов. Жирным шрифтом выделены слова, которые, по мнению авторов, не должны попадать в список самых абстрактных в рамках исследуемого корпуса слов, т.е. ошибочно определённые слова.
\begin{itemize}
    \item \textbf{Betweenness Centrality}: моделирование, модель, структура, оптимизация, математическая модель, математическое моделирование, управление, \textbf{мониторинг}, образование, прогнозирование, эксперимент, \textbf{прочность}, методы, методика, \textbf{самоорганизация}, история, \textbf{адаптация}, \textbf{здоровье}, \textbf{синтез}, анализ, \textbf{эффективность}, свойства, диагностика, \textbf{инновации}, \textbf{оценка}, технология, \textbf{устойчивость}, безопасность, личность, \textbf{надежность}, компьютерное моделирование, \textbf{взаимодействие}, динамика, качество, термодинамика, \textbf{плазма}, \textbf{наночастицы}, развитие, исследование, культура, \textbf{лазер}, теория, интеграция, модернизация, \textbf{деформация}, \textbf{метод конечных элементов}, \textbf{конкурентоспособность}, численное моделирование, \textbf{студенты}, алгоритм.
    \item \textbf{Closeness Centrality}: модель, моделирование, структура, \textbf{оптимизация}, управление, прогнозирование, методика, эксперимент, анализ, математическая модель, методы, математическое моделирование, \textbf{мониторинг}, \textbf{эффективность}, \textbf{надежность}, качество, технологии, \textbf{прочность}, расчет, \textbf{оценка}, планирование, \textbf{инновационная культура}, исследование, инновации, синтез, \textbf{устойчивость}, \textbf{взаимодействие}, образование, проектирование, безопасность, обучение, динамика, свойства, деформация, информационная система, \textbf{самоорганизация}, \textbf{инновационная деятельность}, вероятность, \textbf{профессионализм}, эксплуатация, \textbf{здоровье}, интеграция, инновационное развитие, кинетика, \textbf{температура}, \textbf{вуз}, \textbf{адаптация}, \textbf{работоспособность}, история, алгоритм.
    \item \textbf{Degree Centrality}: моделирование, математическая модель, математическое моделирование, \textbf{оптимизация}, модель, образование, управление, структура, мониторинг, \textbf{личность}, \textbf{прочность}, инновации, свойства, прогнозирование, \textbf{эффективность}, \textbf{синтез}, методика, культура, \textbf{метод конечных элементов}, безопасность, \textbf{оценка}, компьютерное моделирование, \textbf{наночастицы}, \textbf{развитие}, \textbf{адаптация}, эксперимент, \textbf{студенты}, \textbf{здоровье}, качество, история, \textbf{анизотропия}, \textbf{надежность}, технология, \textbf{компетентностный подход}, \textbf{инновационная деятельность}, численное моделирование, диагностика, модернизация, разрушение, \textbf{конкурентоспособность}, творчество, интеграция, \textbf{высшая школа}, \textbf{компетенции}, \textbf{самоорганизация}, \textbf{устойчивость}, динамика, \textbf{вуз}, \textbf{остаточные напряжения}, кинетика.
    \item \textbf{EigenVector Centrality}: образование, управление, модель, инновации, моделирование, \textbf{эффективность}, \textbf{инновационная деятельность}, наука, \textbf{личность}, методика, \textbf{оптимизация}, модернизация, технологии, прогнозирование, мониторинг, компетенции, государство, конкурентоспособность, структура, развитие, интеграция, математическая модель, качество, \textbf{оценка}, анализ, история, высшая школа, культура, взаимодействие, студенты, \textbf{надежность}, инновационное развитие, методы, \textbf{власть}, \textbf{бизнес}, \textbf{вуз}, стратегия, \textbf{компетенция}, эксперимент, \textbf{инновационная культура}, обучение, планирование, \textbf{бакалавриат}, общество, \textbf{компетентностный подход}, \textbf{здоровье}, \textbf{инновационный потенциал}, математическое моделирование, концепция, проект.
    \item \textbf{PageRank Centrality}: моделирование, математическая модель, математическое моделирование, \textbf{оптимизация}, модель, образование, \textbf{мониторинг}, структура, управление, \textbf{метод конечных элементов}, прогнозирование, \textbf{прочность}, \textbf{наночастицы}, компьютерное моделирование, \textbf{личность}, \textbf{эффективность}, инновации, развитие, диагностика, численное моделирование, методика, безопасность, \textbf{компетентностный подход}, культура, \textbf{синтез}, \textbf{адаптация}, свойства, \textbf{здоровье}, \textbf{оценка}, \textbf{устойчивость}, технология, \textbf{надежность}, \textbf{разрушение}, \textbf{наноструктуры}, \textbf{студенты}, интеграция, история, \textbf{роман}, динамика, \textbf{анизотропия}, \textbf{профессиональное образование}, кинетика, алгоритм, \textbf{плазма}, \textbf{вуз}, \textbf{конкурентоспособность}, качество, \textbf{качество образования}, \textbf{остаточные напряжения}, \textbf{дистанционное обучение}.
\end{itemize}

\chapter{Приложение Б. Найденные в коллекции документов тематические теги} \label{AppendixB}
Жирным шрифтом выделены те теги, которые определены верно.

эпр, медь, алтай, \textbf{аудит}, музей, поиск, \textbf{право}, доходы, охрана, смазка, стресс, тьютор, услуги, \textbf{физика}, катализ, матрица, порошок, контекст, покрытия, преграда, адсорбция, \textbf{биометрия}, коррекция, облучение, \textbf{семантика}, \textbf{кинематика}, \textbf{статистика}, предприятие, детали машин, станки с чпу, тестирование, фитопланктон, гидродинамика, дальний восток, самореализация, \textbf{конструирование}, диоксид циркония, жидкие кристаллы, пограничный слой, \textbf{факторный анализ}, \textbf{массовая культура}, преподаватель вуза, имитационная модель, управление знаниями, \textbf{нелинейные колебания}, \textbf{регрессионный анализ}, \textbf{электронное обучение}, ресурсное обеспечение, электроэнцефалограмма, \textbf{оптимальное управление}, \textbf{физическое моделирование}, образовательная программа, образовательные технологии, поддержка принятия решений, высокоскоростное соударение, \textbf{педагогическая деятельность}, международное сотрудничество, научно-образовательный центр, профессиональные компетенции, система менеджмента качества, экспериментальные исследования, \textbf{нелинейные динамические системы}, \textbf{финансово-хозяйственная деятельность}, федеральный государственный образовательный стандарт, nanoparticles.

Некоторые теги не определяют название дисциплины или направления, но по ним также можно понять тематику документа. Поэтому считается разумным отнести к правильно определенным тематическим тегам следующие:

охрана, покрытия, коррекция, облучение, детали машин, дальний восток, самореализация, управление знаниями, ресурсное обеспечение, образовательная программа, образовательные технологии, профессиональные компетенции, система менеджмента качества.

Далее представлены результаты работы программной реализации алгоритма на данных из Веб.

\textbf{trade}, \textbf{testing}, \textbf{principal component analysis}, mechanical properties, microstructure, heterogeneity, identification, globalization, \textbf{semantic web}, turkey, australia, sensors, information, oxidative stress, wireless sensor networks, tracking, \textbf{privacy}, \textbf{sustainable development}, \textbf{architecture}, feature extraction, obesity, apoptosis, conservation, \textbf{pattern recognition}, \textbf{risk assessment}, \textbf{kinetics}, poverty, india, depression, \textbf{cryptography}, climate, diagnosis, virtual reality, parameter estimation, gene expression, collaboration, \textbf{policy}, chaos, detection, finite element method, breast cancer, copper, \textbf{optimal control}, algorithms, mems, memory, decomposition, concrete, xml, usa, corrosion, taxonomy, \textbf{dynamic programming}, planning, volatility, aggregation, \textbf{spectroscopy}, russia, \textbf{dynamics}, density, mobility, dna, \textbf{cfd}, \textbf{sensitivity analysis}.

Аналогично случаю с чистыми данными, можно дополнить список следующими словами:

mechanical properties, microstructure, wireless sensor networks, virtual reality.
