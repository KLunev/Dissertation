\chapter{Приложения моделей близости ключевых слов} \label{chapt1}
В настоящем разделе рассматриваются реальные задачи и системы, в которых применяются программные реализации алгоритмов. 
Рассматривается задача поиска эксперта в области, определяемой заданным запросом из ключевых слов.
Также описывается использование тезауруса ключевых слов для задачи улучшения ранжирования в поисковой составляющей интеллектуальной системы <<ИСТИНА>>.
В конце раздела представлены выводы, а также дальнейшее направление в применении разработанных программных комплексов в реальных приложениях.

\section{Решение задачи поиска экспертов} \label{expert_search}
\subsection{Постановка задачи}
Дано множество наборов ключевых слов $W_X$ и объектов информационной системы $X$, а также множество $Q$ запросов к системе. Обозначим за $W$ множество всех уникальных ключевых слов из всех наборов $W_X$. Каждый элемент $x_i \in X$ множества объектов ассоциирован с набором ключевых слов $W_i = \{w_{i_0}, w_{i_1}, ..., w_{i_{n_i}} \} \in W_X \in 2^W$. Точно также каждый запрос $q_j \in Q$ связан с некоторым набором ключевых слов $W_j = \{w_{j_0}, w_{j_1}, ..., w_{j_{n_j}} \} \in 2^W$. Необходимо определить меру близости пары запрос­объект для каждого объекта и каждого запроса, т.е.  функцию $f : Q \times X \rightarrow R$. Поскольку запросам и объектам единственным образом сопоставляются наборы ключевых слов, то задача сводится к определению меры близости на наборах: $f_w : 2^W \times 2^W \rightarrow R$. Кроме того, необходимо разработать эффективный алгоритм, который, используя меру близости и некоторые дополнительные идеи, мог бы по запросу выдавать множество объектов, наиболее релевантных данному запросу.
\subsection{Процедура поиска экспертов}
По данному множеству наборов ключевых слов (множеству экспертов) строится граф ключевых слов. Далее необходимо для каждого ключевого слова $x$ найти ближайшие по смыслу слова. Мера близости слов вычисляется сначала между тегом $x$ и его соседями в графе, после чего просматриваются соседи соседей $x$ и так до тех пор, пока не наберется фиксированное число кандидат. Часть наиболее релевантных тегов сохраняются, как наиболее близкие к $x$. В дополнение к этому, строится инвертированный индекс, который позволяет по слову восстановить наборы, содержащие это слово. После того, как в систему приходит запрос, для каждого слова из запроса выгружаются ближайшие по смыслу слова и первоначальный запрос расширяется. Затем по словам из расширенного запроса восстанавливаются наборы­кандидаты. Для каждого из них считается мера близости с исходным запросом $TupleSim_{expert}$, подробно описанная в \ref{expert_search_tuplesim}. В конце своей работы алгоритм возвращает наиболее релевантные наборы­кандидаты.
\section{Реализация поиска по ключевым словам на базе собранного тезауруса синонимов}
\subsection{Поисковые подсказки для ключевых слов}
\subsection{Методы расширения поискового запроса}
\section{Выводы}
%\
