\chapter*{Введение}							% Заголовок
\addcontentsline{toc}{chapter}{Введение}	% Добавляем его в оглавление
\nocite{*}
%Современные информационные системы позво
Основными задачами современных информационных систем является эффективная организация хранения, поиска и анализа информации.
На сегодняшний день наибольшие из таких систем способны хранить в себе данныe огромного размера. Стремительный рост объемов информации ведет к необходимости исследования методов и разработки программных комплексов, более эффективно решающих задачи хранения, анализа и поиска данных внутри системы. 
Ключевые слова (или теги) - это набор слов естественного языка или терминов, которые коротко описывают документ информационной системы. Они используются в качестве метаинформации для публикаций (в том числе и научных), что позволяет читателю быстро понять основное направления изложения и концепцию представленной информации, а также указывают некоторые понятия, с помощью которых решаются поставленные задачи. Кроме этого, ключевые слова можно рассматривать как классификаторы контента.

Многие современные информационно-коммуникационные структуры, такие как социальные сети, блоговые и поисковые системы, используют ключевые слова для описания содержащихся в них сущностей (объектов). Такой подход значительно упрощает для пользователя поиск необходимых ему объектов системы, потому что позволяет сделать это с помощью запроса к системе на естественном языке.  Кроме того, ключевые слова помогают поисковым системам по данному запросу выделять наиболее релевантные объекты системы. К числу таких объектов относятся, например, текстовые документы, изображения, видеозаписи и любой другой объект, которому был приписан набор ключевых слов. Многие исследователи активно занимались и продолжают заниматься анализом ключевых слов в целях кластеризации, визуализации, классификации, индексации и поиска целевых объектов.

Исследования, результаты которых представлены в настоящей диссертации, затрагивают важную и востребованную практикой задачу кластеризации объектов по ключевым словам, ассоциированным с этими объектами. Её решение помогает находить в больших информационных коллекциях кластера похожих объектов, удалять дубликаты документов, определять экспертные сообщества. Далее под мерой (степенью) смысловой близости и похожести (далее - <<близость>>, <<схожесть>>) будет подразумеваться показатель семантического сходства пары рассматриваемых ключевых слов или набора слов естественного языка.




\newcommand{\actuality}{}
\newcommand{\progress}{}
\newcommand{\aim}{{\textbf\aimTXT}}
\newcommand{\tasks}{\textbf{\tasksTXT} }
\newcommand{\novelty}{\textbf{\noveltyTXT}}
\newcommand{\influence}{\textbf{\influenceTXT}}
\newcommand{\methods}{\textbf{\methodsTXT}}
\newcommand{\defpositions}{\textbf{\defpositionsTXT}}
\newcommand{\reliability}{\textbf{\reliabilityTXT}}
\newcommand{\probation}{\textbf{\probationTXT}}
\newcommand{\contribution}{\textbf{\contributionTXT}}
\newcommand{\publications}{\textbf{\publicationsTXT}}


{\actuality} Обзор, введение в тему

{\aim} данной работы является \ldots


{\novelty}
\begin{enumerate}
  \item Впервые \ldots
  \item Впервые \ldots
  \item Было выполнено оригинальное исследование \ldots
\end{enumerate}

{\influence} \ldots

{\methods} \ldots

{\probation} \ldots

{\contribution} Автор принимал активное участие \ldots

 % Характеристика работы по структуре во введении и в автореферате не отличается (ГОСТ Р 7.0.11, пункты 5.3.1 и 9.2.1), потому её загружаем из одного и того же внешнего файла, предварительно задав форму выделения некоторым параметрам


\textbf{Объем и структура работы.} Диссертация состоит из~введения, четырёх глав, заключения и~двух приложений.
%% на случай ошибок оставляю исходный кусок на месте, закомментированным
%Полный объём диссертации составляет  \ref*{TotPages}~страницу с~\totalfigures{}~рисунками и~\totaltables{}~таблицами. Список литературы содержит \total{citenum}~наименований.
%
Полный объём диссертации составляет
\formbytotal{TotPages}{страниц}{у}{ы}{}, включая
\formbytotal{totalcount@figure}{рисун}{ок}{ка}{ков} и
\formbytotal{totalcount@table}{таблиц}{у}{ы}{}.   Список литературы содержит  
\formbytotal{citenum}{наименован}{ие}{ия}{ий}.
