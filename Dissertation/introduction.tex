
\chapter*{Введение}							% Заголовок
\addcontentsline{toc}{chapter}{Введение}	% Добавляем его в оглавление
\nocite{*}
Основными задачами современных информационных систем является эффективная организация сбора, хранения, систематизации, поиска и анализа данных.
На настоящее время наиболее представительно и массово востребованные из таких систем способны хранить огромные объемы данных. Стремительный рост хранящейся в них информации приводит к необходимости исследования методов и инструментальных средств разработки программных комплексов, более эффективно решающих задачи организации сбора и хранения, поиска и анализа данных внутри таких больших систем. 

\hl{Исследования, результаты которых представлены в настоящей диссертации, затрагивают важные и востребованные практикой задачи интеллектуального анализа объектов информационно-аналитической наукометрической системы. Автором предлагаются методы решения задачи определения семантической близости объектов, кластеризации объектов, поиска экспертов в различных областях научных знаний, определения тематической направленности объектов. Решение этих задач улучшает качество работы поисковых механизмов, упрощает работу конечного пользователя  с системой, позволяет определять экспертные сообщества и находить коллекции похожих объектов в системе.}

\hl{В рамках проведенных исследований основополагающей является задача определения семантической близости пары объектов. Постановка этой задачи требует определения понятия <<семантически похожих объектов>>. Понятие семантической близости изучалось многими авторами (например, в работах \cite{frawley,sem_DBLP}). Далее под мерой (степенью) смысловой (семантической) близости и похожести (далее - <<близость>>, <<схожесть>>) будет подразумеваться показатель семантического сходства пары рассматриваемых слов или пары наборов слов естественного языка. Здесь следует также отметить, что в контексте рассматриваемой проблемной области мера не всегда является мерой в строгом математическом смысле. В этой сложно формулируемой проблемной области, как правило, активно используются интуитивно понятные эвристические соображения, понятия и основанные на них математические модели и алгоритмы.} 
%%по ключевым словам естественного, ассоциированным с этими объектами,  

Под мерой смысловой схожести в исследовании, результаты которого представлены в настоящей диссертации, будем понимать величину, которая сложно поддается формальному определению. Несмотря на это, интуиция позволяет дать следущее определение паре семантически близких слов: если в речи или в письменном изложении присутствует возможность заменить одно слово на другое так, что смысл предложения не изменится, то два эти слова (заменяемое и замененное) семантически близки. Другими словами, у слушателя возникнет одинаковое представление о цитируемом объекте реального мира в обоих случаях.

Более того, легко дать определение семантически различным словам: после замены одного такого на другое предложение сильно изменяется по смыслу или даже становится абсурдным, то есть теряет какой-либо смысл, даже в том случае, если имеется возможность его <<домыслить>> до некоторого синтаксически корректного предложения \hl{(поставив, например, это слово в подходящую форму).}

Рассмотрим, например, предложение <<пошив мужского костюма>> и последоватьльно будем заменять слово <<костюм>> на слова <<одежда>>, <<фрак>>, <<обувь>>, <<костюмер>>, <<насекомые>>, <<карнавал>>. Если в первых двух случаях заменя кажется разумной: <<пошив мужской одежды>>, <<пошив мужского фрака>>, то третий пример значительно изменяет смысл предложения: <<пошив мужской обуви>>. При замене на четвертое и пятое слова, полученное предложение окончательно теряет смысл.

Исходя из этих соображений, можно всем парам семантически близких слов давать значение меры близости, равное $1$, а всем различным - $0$. Трудность возникает в случаях, когда пара слов не является в рамках данных определений ни парой близких по смыслу, ни парой различных по смыслу слов. Таким парам необходимо ставить некоторое промежуточное значение из интервала $(0, 1)$. В рамках примера, описанного выше, такой парой может являться пара <<костюм-обувь>>. В этот момент возникает неоднозначность в определении того, какое именно значение должна получить данная пара и по какому принципу ранжировать близости различных пар.  Что является более близкими понятиями: пара, связанная отношением гиперонимии (<<стол-мебель>>), пара слов, часто встречающаяся в одних предложениях (<<уголовный-кодекс>>) или <<слова-братья>>, имеющие общего предка-гиперонима (<<декабрь-ноябрь>>)?

Ответ на этот вопрос кроется в постановке задачи, которую мера семантической близости призвана решить. Если рассматривать в качестве системы интернет-магазин и решать внутри этой системе задачу рекомендации товаров, то логично предложить пользователю такой товар, которые часто покупают с тем, что он приобрел. Другими словами, в качестве <<близкого по смыслу>> взять тот, который чаще всего встречается с заданным. Следует однако отметить, что в данном примере предложение товара, абсолютно идентичного по смыслу с уже купленным (тот же самый товар), не имеет смысла.

Другим примером различной трактовки семантической близости может быть классическая поисковая система с программным модулем поисковых расширений. Поисковые расширения - это модуль, добавляющий в текст запроса пользователя новые слова, связанные с запросом. Это позволяет при ранжировании документов вывести на более высокие позиции документы, содержащие эти новые добавленные слова. 

Если эти слова - действительно синонимы  к словам запроса, то скорее всего выдача обогатится и это улучшит ранжирование в целом. Если же расширить слова запроса гиперонимами (например, по запросу <<купить iphone Москва>> для слова <<iphone>> добавить гипероним <<смартфон>>, а для слова <<Москва>> - <<Россия>>, то в выдачу попадут предложения о продаже различных смартфонов (не только iphone), которые, к тому же, будут продаваться не только в Москве, а по всей России. Тем не менее, в некоторых случаях, например, когда задано слово, по которому нет практически никаких документов в базе, добавление документов с гиперонимом к этому слову обычно является разумной идеей.

Однако, самый плохой из возможных сценариев - расширять словами-братьями. В этом случае iphone заменится на (в некотором смысле) близкое по смыслу samsung и пользователь будет весьма опечален: если в прошлой ситуации он попадал на сайты со \emph{смартфонами}, среди которых мог быть нужный ему, то теперь ему целенаправленно показывают на выдаче нерелевантный для него товар.

Кроме того, можно заметить, что уровень близости зависит от тематики той системы, в которой слова употребляются. В системе самой общей тематики слова <<школа-университет>> должны иметь достаточно высокий уровень близости. Если же рассматривать некий образовательный портал, тематическая направленность которого узкоспециальна и относится к образованию, близость данной пары должна быть заметно ниже, потому что в данном контексте это два совершенно разных учебных заведения и различия в данной ситуации имееют принципиальное значение.

Помимо этого, слова могут быть многозначными и даже тривиальная пара <<орган-орган>> может иметь уровень схожести близкий к нулю, если считать, что первое слово - это музыкальный инструмент, а второе - часть тела или термин из юриспруденции (при этом совершенно не очевидно, какое из этих двух значений ближе по смыслу к музыкальному органу).

Описанные выше примеры показывают все неоднозначность трактовки семантической близости на примере слов естественного языка. В рамках данной работы принимается следующее правило: чем слабее изменяется смысл при замене одного слова вторым в различных предложениях, содержащих первое слово, тем больше семантическая близость этой пары слов. 

В этой связи еще раз подчеркивается, что для достижения поставленых в данной диссертации целей, были использованы модели, которые во многом опираются на эвристические, интуитивные понятия. Следует также отметить, что задача разрешения смысловой многозначности в рамках данной работы не рассматривается.

Высоким уровнем близости должны обладать синонимы в привычном значении из языкознания, правильные расшифровки аббревиатур, переводы слова на другие языки, формы одного слова, различные способы написания. В следующей далее таблице \ref{tbl:sim_example} приведены примеры семантически похожих пар ключевых слов из наукометрической системы:
\begin{tabularx}{16cm}{|X|X|X|}
        \hline
        Первое слово & Второе слово & Комментарий \\ \hline
        умение & навык & Синонимия \\ \hline
        полином & многочлен & Синонимия \\ \hline
        β-адреноблокаторы & бета-адреноблокаторы & Различные способы написания одного слова \\ \hline
        орви & острые респираторные вирусные инфекции & Расшифровка аббревиатуры \\ \hline
        хехцир & khekhtsyr & Транслитерация \\ \hline
        корень & корни & Разные формы одного слова \\ \hline
        crisis & кризис & Перевод на другой язык \\ \hline
        cu & медь & Другая форма названия \\ \hline
\caption{Примеры семантически близких ключевых слов} \label{tbl:tuple_test}
        \label{tbl:sim_example}
\end{tabularx}

Одним из известных и широкоиспользуемых способов высокоуровневого описания данных, представленных в системе, является использование \emph{ключевых слов}. Ключевые слова (или теги) - это набор слов естественного языка или терминов, которые коротко описывают отдельный документ, который хранится в информационной системе. Они используются в качестве метаинформации для публикаций (в том числе и научных) в средствах массовой информации и печатных изданий. Такой подход позволяет читателю быстро понять основное направления изложения и концептуальные положения представленной информации, отмечают некоторые понятия и сущности, с помощью которых решаются представленные в этих публикациях задачи. 

Многие современные информационно-коммуникационные структуры, такие как социальные сети, блоговые и поисковые системы, используют ключевые слова для описания содержащихся в них сущностей (объектов). Такой подход значительно упрощает для пользователя поиск необходимых ему объектов системы, потому что позволяет сделать это с помощью запроса к системе на естественном языке. Кроме того, ключевые слова помогают поисковым системам по данному запросу выделять наиболее релевантные объекты системы. К числу таких объектов относятся, например, текстовые документы, изображения, видеозаписи и любой другой объект, которому был приписан набор ключевых слов. Многие исследователи активно занимались и продолжают заниматься анализом ключевых слов в целях кластеризации, визуализации, классификации, индексации и поиска целевых объектов.

Кроме того, ключевые слова можно рассматривать и как классификаторы контента, формирующие тезаурус предметной области, на основе которого этот контент описывается. Примером такого классификатора является \emph{универсальная десятичная классификация (УДК)}, используемая для систематизации и группировки накопленных человечеством знаний по тематическим разделам. Данная классификация различным областям науки, литературы и искусства ставит в соответсвии цифровые коды. Описание областей задается с помощью небольшого набора ключевых слов, характеризующих данное направление. Данные УДК построены по иерархическому принципу: более общие направления науки (а также соответствующием им коды) описываются общими по смыслу словами, например, <<Общественные науки>>. При углублении и выборе определенной специализации внутри данного направления, описание приводится с помощью более конкретных понятий, таких как <<Политика>>, <<Право>>, <<Экономика>>, <<Народное хозяйство>> и т.д.. Таким образом, с помощью небольшого множества ключевых слов появляется возможность структуризации необходимого любого направления и соответсвующего ему кода. Несмотря на все многообразие тематик, поиск необходимого кода не занимает много времени, что является возможным благодаря использованию ключевых слов.

\hl{Важным является то обстоятельство, что реальные информационно-аналитические системы во многих случаях не обладают достаточным объемом данных для анализа. Рассмотрим, например, научные публикации, как объекты наукометрической системы. Зачастую в таких данных отсутствует полнотекстовая информация. Доступной в этом случае является лишь метаинформация: авторы, название, ключевые слова. В связи с этим, в рамках проводимых исследований информация об объекте ограничивается набором ключевых слов на естественном языке, а также связями данного объекта с другими объектами системы. Другими словами, каждому объекту системы ставится в соответствии набор ключевых слов. Семантическая близость между объектами такой системы сводится к семантической близости между соответствующими им наборами ключевых слов. В свою очередь, семантическая близость между парой наборов ключевых слов опирается на семантическую близость между словами, входящими в эти наборы.} 

\hl{Кроме того, отмечается, что каждый объект описывается очень малым объемом данных (обычно это 5-6 слов). Это обстоятельство вносит существенные ограничения в допустимые методы решения обозначенных выше задач.}




\newcommand{\actuality}{}
\newcommand{\progress}{}
\newcommand{\aim}{{\textbf\aimTXT}}
\newcommand{\tasks}{\textbf{\tasksTXT} }
\newcommand{\novelty}{\textbf{\noveltyTXT}}
\newcommand{\influence}{\textbf{\influenceTXT}}
\newcommand{\methods}{\textbf{\methodsTXT}}
\newcommand{\defpositions}{\textbf{\defpositionsTXT}}
\newcommand{\reliability}{\textbf{\reliabilityTXT}}
\newcommand{\probation}{\textbf{\probationTXT}}
\newcommand{\contribution}{\textbf{\contributionTXT}}
\newcommand{\publications}{\textbf{\publicationsTXT}}
\newcommand{\workrequirements}{\textbf{\workrequirementsTXT}}


{\actuality} В качестве предмета исследования и анализа в диссертации выступают объекты наукометрической информационно-аналитической системы, которые описываются наборами ключевых слов. Кроме того, объекты такой системы связаны между собой различными отношениями, например, для научной публикации это может быть список соавторов, для научного работника - список проектов, в выполнении которых он принимал участие, или список конференций, в которых он принимал участие. Публикации, персоналии, научные проекты и конференции в данном примере являются объектами и, следовательно, могут иметь собственные наборы ключевых слов. 

Побудительным мотивом и конечной целью исследований, результаты которых представлены в настоящей диссертации, является создание интеллектуального программного модуля, встраиваемого в наукометрическую информационно-аналитическую систему, способного по имеющимся в системе ключевым словам и определенным связям между ними выявлять семантическую информацию и с ее помощью решать задачи информационного поиска и классификации. Следует также отметить то обстоятельство, что зачастую информационные системы не обладают большим объемом данных для анализа, что делает затруднительным качественное семантическое сравнение объектов. Однако и в таких системах необходимо уметь точно определять релевантную пользователю информацию. Как следствие, важным требованием к разрабатываемому модулю является его способность эффективно работать в условиях ограниченного объема входной информации.

Кроме того, отсутствие достаточного объема данных в реальных системах показывает актуальность и востребованность на практике исследования, результаты которого представлены в настоящей диссертации.

{\aim}  диссертационной работы является исследование и разработка математических моделей, алгоритмов и программных средств интеллектуального анализа наборов ключевых слов, характеризующих объекты в наукометрических интеллектуальных системах, с использованием методов из теории графов и дополнительной информации онтологического характера об объектах в системе. Такая деятельность соответствует областям исследования, отмеченным в пп. 1, 2, 5, 9 Паспорта специальности 05.13.17 – теоретические основы информатики.

{\workrequirements}
Согласно стандарту \mbox{ГОСТ Р ИСО/МЭК 9126-93} к качеству разрабатываемой системы интеллектуального анализа объектов информационной системы предъявляются следующие требования:
\begin{itemize}
    \item широкие функциональные возможности; 
    \item надежность;
    \item эргономичность;
    \item эффективность;
    \item сопровождаемость;
    \item мобильность.
\end{itemize}

Более подробно изложенные выше пункты определены в приложении {\ref{AppendixRequirements}}. Описанные в данном приложении характеристики определяют отличительные стороны решаемой в настоящей диссертации задаче от известных работ, связанных с выделением семантической информации между объектами информационных систем. Существующие системы в большинстве своем опираются на обилие входных данных, к числу которых относятся:
\begin{itemize}
    \item текстовая информация, а именно - аннотации, заголовки, полные тексты документов;
    \item общие объемы системы, которые характеризуются значительным количеством сущностей внутри системы и число связей между ними.
\end{itemize}

В то же время, разрабатываемый программный комплекс является более гибким решением для систем, не обладающим большим объемом данных. Такие системы с одной стороны не содержат в себе огромного количества различных объектов. С другой стороны, о каждом из объектов известно минимальное количество информации - сущности таких систем должны лишь обладать описывающим их набором ключевых слов, либо быть соединены внутренними связями с сущностями, которым набор ключевых слов ассоциирован. Кроме того, разработанные подходы позволяют получать узконаправленные семантические модели для конкретной области знаний. Ручной труд при внедрении таких систем сводится к минимуму.

В разделе {\ref{related_work_concl}} содержится краткое изложение мотивации предъявленных требований к системе, разрабатываемой в данной работе. Кроме того, представлены недостатски существующих методов решения подобных задач. Описываются проблемные места, которые не позволяют применять эти подходы к некоторому классу систем. В конечном итоге выделяется \textbf{специфика} разрабатываемого комплекса, отличающего его от известных аналогов. На основе анализа перечисленных требований была разработана методология решения поставленной задачи.

{\methods} включает следующие характеризующие ее аспекты.
\begin{itemize}
    \item \textbf{Концептуальные положения.} 
        \begin{itemize}
            \item Опора на наборы ключевых слов, ассоциированых с объектами информационной системы.
            \item Возможность использования различных информационных объектов (источников) - НИР, публикации, патенты и т.п..
            \item Наличие механизмов, позволяющих в автоматизированном режиме проводить экспертные оценки адекватности полученных решений.
        \end{itemize}
    \item \textbf{Модели, методы и средства достижения цели.}
        \begin{itemize}
            \item Модели, реализующие концептуальные положения создания и развития на основе графовых представлений данных и эвристических алгоритмов над ними, работающие в условиях отсутствия строгого математического описания.
            \item Методы машинного обучения, необходимые для улучшения качества определения семантических связей между объектами информационной системы;
        \end{itemize}
    \item \textbf{Инструментальные средства.} Для разработки программных комплексов, решающих поставленные задачи, использованы открытые математические, графовые библиотеки, программные пакеты для обработки естественного языка, программные реализации моделей машинного обучения с отрытым исходным кодом.
    %\item \textbf{Архитектурно-технологические решения.}
    \item \textbf{Перечень и постановка задач, решение которых обеспечивает достижение цели.}
        \begin{enumerate}
        \item \hl{Разработка графовой модели представления данных.} Необходимо представить данные системы в виде множества графов, вершинами которых являются некоторые понятия (ключевые слова/наборы слов/сущности системы), а ребрами - отношения между ними. Такие графы необходимы для вычисления различных характеристик для пар понятий. Решение задачи приводится в главе \ref{chapt_word_similarity}.
        \item \hl{Разработка моделей} определения семантической близости пары ключевых слов. Для этого используются построенные графы, разработанные подходы и технологии машинного обучения. \hl{Формируется набор количественных признаков и решающее правило, определяющее степень семантической близости по этому набору.}. Описание разработанных моделей приводится в главе {\ref{chapt_word_similarity}}.
        \item \hl{Разработка моделей} определения семантической близости пары наборов ключевых слов. Разработанные модели используют различные графовые представления, подходы и модели, рассмотренные в предыдущих пунктах. Решению этой задачи посвящена глава \ref{chapt_tuple_similarity}.
        \item \hl{Апробация разработанных моделей.} Используя функцию близости наборов ключевых слов и отношения между сущностями системы, решаются прикладные задачи определения семантической близости пары сущностей. Этой задаче посвящается глава \ref{chapt_applications}.
        \end{enumerate}
\end{itemize}

Для достижения поставленных целей и удовлетворения описанных выше требований были рассмотрены различные методы решения, их преимущества и недостатки. По окончании поиска была составлена методология исследования,  наиболее подходящая поставленным в настоящей диссертации задачам в рамках имеющихся особенностей и ограничений в наборов исходных данных. Подробная мотивация выбранной методологии описывается в \ref{methodology}.

%Первым важным решением было использование графовой схемы представления данных. При этом во внимание принимались следующие аргументы:
%\begin{itemize}
%    \item в связи с тем, что информационная система может не располагать большими объемами данных, важно в полной мере использовать каждый объект и каждую связь в системе;
%    \item для этого подходящим является графовое представление данных, поскольку по графам можно восстановить частичную недостаточность данных. Отсутствие реальной связи между объектами в силу незаполненности системы можно восполнить, проанализировав пути и отношения в графе;
%    \item в следствие небольших объемов входных данных, графовое представление является эффективным способом хранения информации;
%    \item между сущностями возможны разные типы отношений. Построение одного графа для нескольких таких типов может позволить эффективно пользоваться взаимодействием между этими отношениями;
%    \item графовое представление открывает огромные возможности по подсчету различных характеристик для пары сущностей, оцениваемых на уровень похожести. Появляется возможность подсчета пути, количеств различных путей, мер центральностей, соседств, потоков, различных графовых индексов, кластерного анализа  и использования всего богатого инструментария теории графов. Все это может оказать положительный эффект на качество определения семантической близости.
%\end{itemize}

%. То есть сначала решаются базовые задачи определения близости пары слов, затем процесс поднимается на более высокий уровень определения близости наборов, после чего происходит переход к поставленным в диссертации задачам. Мотивировка данного решения излагается далее.


В работе применяются методы анализа текстов на естественном языке, методы машинного обучения и программной инженерии. При изложении результатов диссертационной работы широко используется аппарат теории графов, а также математической логики и математической статистики.

{\novelty}
работы определяется тем, что автором разработаны новые алгоритмы определения семантической близости для пары ключевых слов, а также для пары наборов ключевых слов, описывающих объекты интеллектуальной наукометрической системы. Созданы уникальные методы автоматической генерации обучающей выборки, а также методы автоматической проверки качества работы программных реализаций алгоритмов определения семантически похожих ключевых слов и алгоритмов выявления кластеров близких понятий. Последнее обстоятельство важно, поскольку тестирование программ в данной предметной области требовательно к наличию специалистов, способных точно определить степень близости для пары объектов или понятий. Разработаны алгоритмы построения иерархических классификаторов научных направлений в автоматическом режиме, использующие исключительно наборы ключевых слов. Важными особенностями  указанных алгоритмов являются: отсутствие необходимости больших объемов данных для обучения моделей с приемлемым уровнем качества; возможность использования разработанных моделей для произвольных интеллектуальных систем, использующих ключевые слова для описания сущностей; небольшие человеческие трудозатраты для выставления экспертных оценок. Проведена работа по уменьшению числа параметров системы, что делает разработанные модели и программные средства эргономичными и легкими для настройки.

На основе исследованний о семантической похожести ключевых и дополнительной информации онтологического характера об объектах, подлежащих анализу, решен ряд значимых задачи и востребованных на практике задач.

Доказана вычислительная сложность разработанных алгоритмов, подтверждающая их адекватность (соответствие) требованиям, предъявляемым к разрабатываемому программному комплексу.

{\influence} Рассматриваемый в работе программный комплекс для анализа, обработки и поиска объектов интеллектуальных информационных систем по ключевым словам представляет собой самостоятельный инновационный продукт. Он может использоваться не только в системе, рассматриваемой в данной диссертации, но и в любой информационно-аналитической системе, объекты которой описываются наборами ключевых слов. Кроме того, разработанные методики обработки связей между объектами могут быть перенесены на другие задачи анализа взаимосвязанных объектов. Рассматриваемый программный модуль определения семантической близости между словами  порождает словарь синонимов той области, на которой был обучен. Этот словарь может быть использован в самых разнообразных задачах информационного поиска и обработки естественного языка, и потенциально может привнести дополнительный полезный сигнал для моделей классификации, ранжирования и кластеризации текстовых или текстово-аннотированных объектов.


\textbf{На защиту выносятся} следующие положения:
\begin{itemize}
    \item методы решения задачи определения семантической близости между парой понятий интеллектуальной, использующие ассоциированные с понятиями ключевые слова, а также  и различными отношениями между этими понятиями;
    \item обоснование актуальности решаемой задачи;
    \item методология, принятая для исследования;
    \item научная новизна и практическая значимость диссертации;
    \item результаты исследований, полный перечень которых приведен в заключении.
\end{itemize}

{\probation} Результаты диссертации докладывались на всероссийской конференции с международным участием <<Знания–Онтологии–Теории (ЗОНТ-2016)>>, на международной конференции <<Ломоносовские чтения>> (2014, 2016, 2018),  на механико-математическом факультете МГУ имени М.В. Ломоносова на семинаре «Проблемы современных информационно-вычислительных систем» под руководством д.ф.-м.н., проф. В.А. Васенина (2013, 2015, 2017, 2018), на семинаре \emph{(не нашел название семинара)} в ВЦ РАН под руководством д.ф.-м.н., проф. В.А.Серебрякова (2018).

По теме диссертации опубликовано 7 работ, в том числе одна в зарубежном издании. Четыре статьи \cite{lunev_paper_1,lunev_paper_2,lunev_paper_3,lunev_paper_4} опубликованы в изданиях из перечня ВАК.
%{\contribution} Автор принимал активное участие \ldots

 % Характеристика работы по структуре во введении и в автореферате не отличается (ГОСТ Р 7.0.11, пункты 5.3.1 и 9.2.1), потому её загружаем из одного и того же внешнего файла, предварительно задав форму выделения некоторым параметрам


\textbf{Объем и структура работы.} Диссертация состоит из~введения, четырёх глав, заключения и~двух приложений.
%% на случай ошибок оставляю исходный кусок на месте, закомментированным
%Полный объём диссертации составляет  \ref*{TotPages}~страницу с~\totalfigures{}~рисунками и~\totaltables{}~таблицами. Список литературы содержит \total{citenum}~наименований.
%
Полный объём диссертации составляет
\formbytotal{TotPages}{страниц}{у}{ы}{}, включая
\formbytotal{totalcount@figure}{рисун}{ок}{ка}{ков} и
\formbytotal{totalcount@table}{таблиц}{у}{ы}{}.   Список литературы содержит  
\formbytotal{citenum}{наименован}{ие}{ия}{ий}.
