\chapter*{Введение}							% Заголовок
\addcontentsline{toc}{chapter}{Введение}	% Добавляем его в оглавление
\nocite{*}
Многие современные информационные системы, такие как социальные сети, блоговое и поисковые системы,  используют ключевые слова для описания содержащихся в них сущностей. Это значительно упрощает для пользователя поиск объектов системы, потому что позволяет с помощью запроса на естественном языке находить документы различной природы: текстовые документы, изображения, видеозаписи - любой объект, которому был приписан набор ключевых слов. Многие исследователи занимались задачами анализа ключевых слов в областях  кластеризации, визуализации и классификации,  индексации и поиска похожих объектов.


% близость слов
В рамках поиска подходов к решению рассматриваемой задачи проведены поисковые исследования. 


Одной из первых метрик похожести двух слов в компьютерных науках является расстояние Левенштейна (редакторское расстояние, \cite{leven}). Это метрика подсчитывает сколько необходимо произвести добавлений, удалений или замен одного символа на другой, чтобы из одной строки получить вторую. Традиционно этот алгоритм используется для исправления опечаток: для введенного слова можно найти ближайшие по этой метрике слова из фиксированного словаря. Существует ряд более сложных версий алгоритма, в числе которых алгоритм Демерау-Левенштейна \cite{leven_dem}. Усовершенствование этого алгоритма заключается в том, что дополнительно используется четвертая операция транспозиции двух соседних символов. Более продвинутые версии алгоритма используют настроенную таблицу стоимостей каждой из операции. Авторы \cite{learn_leven} с помощью разработанных алгоритмов и обучающей выборки определяют стоимость замены одного символа на другой (а также добавления и удаления каждого символа). Авторы высказывают гипотезу о том, что различные замены символов не должны иметь один и тот же вес: если человек опечатался, то вероятно, что он ввел символ, который находится близко к правильному символу на клавиатуре. Такая замена не должна сильно влиять на общее расстояние. Другим примером важности дифференцированного взвешивания замен могут являться безударные гласные: люди чаще путают при написании пару букв <<а>> и <<о>>, чем, например, пару букв <<а>> и <<е>>. Важным достоинством такого алгоритма является то, что слово и его транслитерированная версия (например, <<компьютер>>-<<computer>>) становятся близки по данному расстоянию. Недостаток алгоритма в том, что он требует обучающую коллекцию различных написаний одного слова.
Следующий важный этап развития идеи редакторского расстояния заключается в использовании контекста. В работе \cite{context_leven} авторы настраивают стоимости переходов между символами с учетом контекста. Выдвигается гипотеза о том, что стоимость замены одного символа на другого может сильно зависеть от символов, которые окружают эти два символа. Например, удаление символа <<ь>> более обоснованно в конце глаголов, так как ошибки <<ться>>/<<тся>> популярны и вероятнее всего речь идет об одном и том же слове. Как и в обозначенной выше работе, данный алгоритм требует обучающую выборку, но в данном случае ее размер должен быть значительно больше, поскольку число параметров растет экспоненциально с увеличением значения контекста.

%Другая идея в области использования мер близости на строках для определения смысловой близости -- использование 
Другим направлением исследований в области определения смысловой близости пары слов является использование фонетической информации рассматриваемых слов. Примером таких работ могут служить \cite{soundex,phone_sim}. Работы опираются на гипотезу о том, что похожие слова могут звучать одинаково. Авторы первой работы по слову строят его короткий код таким образом, чтобы различные слова с одним кодом звучали похоже. Авторы второй работы предлагают различные алгоритмы определения фонетической близости, в том числе они используют расстояние Левенштейна на фонемах для рассматриваемых слов.
%, авторы которых при использовании мер близости на строках для определения смысловой близости между понятиями

Базовыми методами определения близости пары понятий является сбор информации о совместной встречаемости слов внутри одного набора. Факт появления пары слов в одном предложении или тексте может быть важным сигналом для определения смысловой близости. Данные методы разбираются в \cite{freq_1,freq_2,pmi}. Более продвинутые алгоритмы основаны на вычислении взаимной информации (Pointwise mutual information), введенной авторами \cite{pmi}, которая выносит решение об уровне близости по совместной встречаемости пары слов в документах, но пессимизирует значение, если слова, для которых вычисляется близость, встречаются слишком часто в представленных данных по-отдельности. Таким образом, высокое значение семантической близости имеют пары слов, которые часто встречаются вместе и редко поодиночке. Для того, чтобы такая мера адекватно представляла близость между понятиями, необходимы корпусы огромных значений: чем меньше раз понятия встретились в документах, тем сильнее каждый отдельный случай совместного появления влияет на данную метрику. Зачастую эта особенность приводит к тому, что наиболее близкими парами в смысле этой меры близости являются те, которые встретились единственный раз в одном общем документе. Для того, чтобы уменьшить негативный эффект на практике, принято исключать пары слов, которые встретились в корпусе меньше некоторого порогового значения. Другой способ обойти сложившуюся трудность в ином определении вероятностей появления каждого из слов, а также вероятности их совместного появления. Для этого служат методики сглаживания вероятности, принятые в области построения языковых моделей. Основные способы сглаживания представлены в работе \cite{lm}. Также существуют усовершенствования PMI меры различными эвристическими предположениями: усредненная и средневзвешенная взаимная информация (average and weighted average mutual information), рассмотренные, соответственно, в \cite{avg_pmi} и \cite{w_avg_pmi}; контекстная усредненная взаимная информация (contextual average mutual information), введенная в \cite{context_pmi}; нормированная взаимная информация (normalized mutual information), введенная в \cite{npmi}, квадратичная и кубическая взаимная информация (PMI2 и PMI3), рассмотренные в \cite{pmi23}. 
В другой работе \cite{search_eng} вопрос вычисления семантической близости решается с помощью поисковых систем. Программа запрашивает пару сравниваемых слов через открытый API и получает совместную и индивидуальные частоты встречаемостей слов в интернете. На основе этой информации подсчитывается уровень похожести слов друг на друга. Очевидным недостатком такого подхода является ограниченная поисковой системой пропускная способность (количество запросов в единицу времени) и общее количество запрос.
Различные вариации PMI-метрик являются, по сути, вероятностными методами, поскольку подразумевают вычисления оценки вероятности встретить каждое из понятий, а также эту пару понятий совместно внутри одного текста. Существуют и другие вероятностные методы сравнения пары слов естественного языка. Например, может быть использован $\chi^2$ критерий и тест отношения правдоподобия. Способ применения данных методов описан в \cite{freq_est_overview}
Важной особенностью в применении PMI-подобных метрик к набору текстов является то, что они не показывают смысловую близость между понятиями в явном виде, а скорее определяют коллокации: <<Российская Федерация>>, <<крейсер Аврора>>, <<завод имени Кирова>>, <<средний класс>>, <<пластическая операция>> и другие. Тем не менее, знание того, что совместная встречаемость пары понятий внутри одного множества слов (предложение, документ, набор ключевых слов, короткое описание объекта и т.д.) статистически значимо превосходит случаи их отдельных появлений, является важным фактором для определения в том числе и семантической близости между этими понятиями.

Многие подходы к решению задачи определения близости пары слов используют понятие нграммы.
Cимвольной/пословной нграммой называют последовательность фиксированной длины из определенного числа подряд идущих символов/слов. Символьные и пословные нграммы широко используются в различных задачах из области обработки естественного языка таких как построение языковых моделей \cite{ngrams_1,ngrams_2,ngrams_3,ngrams_4} и моделей машинного перевода \cite{ngrams_mt_1,ngrams_mt_2,ngrams_mt_3}. Недостатком нграммных моделей является так называемое проклятие размерности, которое в данном случае говорит о том, что при увеличении длины нграммы катастрофически быстро растет число возможных нграмм данного размера, а также параметров системы, что делает затруднительным их применение во многих случаях. Также при работе с нграммами необходимы объемы текстов огромных размеров. Если предметная область, в которой решается задача, является узкоспециальной, то получение данных достаточного объема зачастую является невозможным.

В работе \cite{ngrams_sim} авторами предложены различные метрики близости на основе нграммного представления слов.

В работе \cite{Albatineh2011} авторы используют  меру Жаккара для определение близости пары ключевых слов: каждому ключевому слову ставится в соответствии множество понятий и для определения близости вычисляются размеры объединения и пересечения этих множеств. Отмечается, что как только словам в однозначное соответствие поставлены некоторые множества, сразу становится возможным вычисление близости на основании различных мер близости пары множеств. Помимо меры Жаккара, существует чуть менее популярная мера Серенсена (\cite{dice_1}). Эти и другие меры близости на множествах подробно описаны в \cite{dist_between_sets}.

Авторы \cite{Shirude} для определения близости используют комбинацию три модели определения близости: нграммную модель, модель близости жаккара, а также модель векторного пространства. Последняя подразумевает представление слов в виде вектора определенной длины. Близость в свою очередь сводится к величине скалярного произведения векторов для двух слов. Эта модель детально описывается в \cite{vector_space}. Имея три функции близости, авторы вычисляют среднее по их значениям. Это приводит к тому, что такая композиция уменьшает влияние слабых сторон каждого отдельного алгоритма. Тем не менее такой наивный метод комбинирования может даже ухудшать результат, если входящие в нее модели демонстрируют низкое качество.

Важной особенностью алгоритмов, основанных на вычислении символьной нграммной близости, расстояния левенштейна и наибольшей общей подпоследовательности, является то, что такие методы не дают представления о смысловой близости между словами и способны определять близкие слова только по похожести написания. Это является серьезным недостатком для задач, в которых важна смысловая близость между понятиями. Примером такой задачи может быть разработка классической поисковой системы, где удачное добавление синонимов для слов запроса, сформулированного пользователем, может вылиться в более релевантную выдачу для этого запроса.  Несмотря на этот недостаток, данные методы могут быть применены внутри более сложных алгоритмов для повышения их качества.

Существует множество методов, использующих для вычисления близости внешние наборы данных. Такими наборами могут быть словари, семантические сети, тезаурусы или данные сети Веб. Важным источником знаний об отношениях между словами английского языка является семантическая сеть WordNet (\cite{wordnet}). Слова в данной сети могут быть связаны одним из нескольких отношений: гипероним, гипоним, <<имеет участника>> (факультет-профессор), <<является участником>> (пилот-экипаж), мероним, антоним. Также имеются лексические, антонимические, контекстные связи между словами. Для русского языка существует несколько аналогов: RussNet (\cite{russnet}), YARN (\cite{yarn, yarn_2}), RuThes (\cite{ruthes}), Russian WordNet (\cite{russian_wordnet}). Недостаток этих тезаурусов в их неполноте, а также в том, что некоторые из них не являются публично доступными. Чтобы посчитать близость по таким тезаурусам, строится дерево, в вершинах которого стоят слова (вершина в \textsc{WordNet} является синсетом - множеством слов, не отличимых по смыслу), а ребра указывают на отношение гиперонимии между парой вершин. Таким образом, в листьях дерева лежат узкоспециальные понятия, которые обобщаются их предками в дереве, в корне же лежит слово наиболее общего значения. Имея такое дерево, появляется возможность вычислять смысловую близость понятий по их взаимному расположению внутри этого дерева. Так авторы \cite{wordnet_sim_0} в своей формуле близости используют глубину наиболее конкретного по значению предка, а авторы  \cite{wordnet_sim_1} в дополнении к этому считают расстояние между вершинами. Чем больше расстояние между словами и чем глубже находится общий предок, тем меньше уровень смысловой похожести. В работах \cite{wordnet_hybrid_1,wordnet_hybrid_2} используются гибридные методы определения близости: расстояния и глубина в дереве, вероятности встречаемостей в корпусах, признаки, основанные на свойствах слов в рассматриваемых тезаурусах.

Еще одним открытым источником отношений между понятиями является интернет-энциклопедия wikipedia (www.wikipedia.org). Данная база позволяет эффективно использовать категории, ссылки, полные тексты и мета-данные статей для извлечения семантической информации о словах. Авторы \cite{wiki} строят различные меры близости, опираясь на тексты статей и представляя сравниваемые понятия в виде векторов определенной длины. В \cite{wiki_2} автор использует данные википедии (в частности используются информация о ссылках между статьями) и разработанные им метрики близости слов для решения задачи снятия лексической неоднозначности. 

С развитием вычислительной техники все большей и большей популярностью начинают пользоваться методы, основанные на обучении нейронный сетей. Одним из самых известных методов определения семантической близости является модель \textsc{word2vec} (\cite{word2vec}). Данная модель представляет собой нейронную сеть, на вход которой подаются огромные корпуса текстовых данных. Задачей обучения является построение такого векторного представления для текущего слова (\textsc{word embeddings}), которое максимально точно способно предсказать рядом стоящие в тексте слова. Обученная модель строит векторное пространство, обладающее рядом полезных свойств, которые в наше время широко используются для решения многих задач естественного языка, связанных с семантической информацией. Одним из таких свойств - семантическая близость понятий, векторные представления которых похожи. Таким образом любую пару слов из словаря можно сравнить, использую, например, косинусное расстояние между векторами.

Мощной моделью построения векторных представлений для слов является модель GloVe, которая строит матрицу частотностей встречаемостей слов во всех возможных контекстах. Далее используются методы уменьшения размерности пространства, которые оставляют только наиболее значимые компоненты в разложении. В то время, как \textsc{Word2Vec} является предиктивной моделью, \textsc{GloVe} представляет собой модель на основе подсчета статистики.

Различные методы векторного представления описаны и протестированы на открытых источниках в работе \cite{embed_1}. Несмотря на высокое качество определения семантической близости моделей, использование полнотекстовой информации существенно ограничивает область применения данных методов, посколько для многих прикладных задач не имеется достаточно много текстовых данных. Возникает трудность при работе в узкоспециальных областях: модели, обученные на корпусах общего назначения, не могут улавливать особенности таких областей. Использование текстовых данных из рассматриваемой области для обучения ведет к неправильной настройке параметров модели и недообучению, по причине недостатка этих самых данных. Это выливается в низкий уровень качества моделей. 

При исследовании области семантической близости пары ключевых слов возникает дополнительная информация о наборах, в которые входят рассматриваемые слова. Помимо этого зачастую для набора известен также объект, к которому этот набор приписан. Авторы \cite{folk} вводят понятие фолксономии. Фолксономией называется кортеж $(U, T, R, Y)$, где $U,T$ и $R$ - конечные множества, элементами которых служат, соответственно, пользователи, ключевые слова и ресурсы. $Y$ - тернарное отношение между ними, т.е. $Y  \subseteq U \times T \times R$. Постом называется тройка $(u, T_{ur}, r)$, где $u \in U, r \in R$, $T_{ur}$ - непустое множество ключевых слов такое, что $T_ur  \coloneqq {t \in T | (u, t, r) \in Y}$. Авторы \cite{folk_2} считают близость между ключевыми словами несколькими способами. Первый способ заключается в построении меры близости по статистике совместной встречаемости пары ключевых слов. Второй способ предполагает построение векторного пространства для каждого слова. На $i-$ой позиции стоит количество документов, в которые одновременно входит рассматриваемое ключевое слово и $i-$ое. Далее мера близости вводится как косинусное расстояние между векторами в этом пространстве. Последний способ, который описывается в \cite{folk} подсчитывает меру, подобную мере \textsc{PageRank} (\cite{pagerank}) для документов в сети Веб.
   В [PAPER] авторы проводят вычисление близости, основываясь на теоретико-графовых алгоритмах. Рассматривается граф, вершинами которого являются ключевые слова, а ребра показывают факт принадлежности пары слов одному набору. Далее по построенному графу для пары вершин  вычисляются различные характеристики: расстояние, количество кратчайших путей, различные графовые меры близости. Недостатком описанной в статье модели является тот факт, что смысловая близость между парой слов стремительно падает при увеличении расстояния в графе. Это происходит по той причине, что набор ключевых слов далеко не всегда состоит из близких по смыслу слов. Поэтому при переходе от одной вершины к другой, вероятность исходной вершины быть близкой по смыслу к другой стремительно уменьшается. В работе [word2vec] для определения контекстной близости между словами естественного языка авторы обучают по корпусу текстов нейронную сеть, представляющую каждое слово в виде вектора не очень большой длины. Имея такое представление, уровень близости пары слов может быть вычислен, как мера близости между векторами, например, с помощью косинусной меры. В работах [] представлены подходы решения задачи с помощью классических методов машинного обучения с учителем. По паре текстов вычисляются вручную разработанные факторы, которые, по мнению авторов, сильнее всего влияют на уровень близости. После чего на этих факторах и обучающем множестве пар документов тренируется модель машинного обучения. В более старых работах близость [] вычисляется при помощи статистической меры TFIDF. Каждому слову документа в соответствии ставится число, которое тем больше, чем чаще это слово встречается в данном документе и реже в других документах. 


%близость наборов
Существующие методы имеют ряд недостатков по отношению к решаемой в данной работе задаче. Основная из них - отстутствие наборов данных достаточного объема. Для качественного обучения нейронной сети необходимы миллионы примеров полноценных текстов, в то время как наборы ключевых слов, как правило, состоят лишь из нескольких слов. Доступ к ресурсам поисковых систем является ограниченным и не имея постоянного доступа к ним, сложно получить хорошие результаты. Сложность применения машинного обучения к такого рода задачам - в отсутствии достаточно больших обучающих выборок для тренировки.

В рамках данной работы представлены методы определения близости по корпусу наборов ключевых слов, также опирающиеся на методы из теории графов. Значительным улучшением является построение второго графа ключевых слов, основанного на контекстной близости пары слов. В следующих далее разделах дано определение контекстной близости для пары ключевых слов, а также представлены методы построения такого графа. После чего показан алгоритмы семантической кластеризации, основанные на введенной мере близости слов. В разделе [?] представлены тестовые данные, результаты экспериментов программных реализаций алгоритмов.


\newcommand{\actuality}{}
\newcommand{\progress}{}
\newcommand{\aim}{{\textbf\aimTXT}}
\newcommand{\tasks}{\textbf{\tasksTXT}}
\newcommand{\novelty}{\textbf{\noveltyTXT}}
\newcommand{\influence}{\textbf{\influenceTXT}}
\newcommand{\methods}{\textbf{\methodsTXT}}
\newcommand{\defpositions}{\textbf{\defpositionsTXT}}
\newcommand{\reliability}{\textbf{\reliabilityTXT}}
\newcommand{\probation}{\textbf{\probationTXT}}
\newcommand{\contribution}{\textbf{\contributionTXT}}
\newcommand{\publications}{\textbf{\publicationsTXT}}


{\actuality} В качестве предмета исследования и анализа в диссертации выступают объекты наукометрической информационно-аналитической системы, которые описываются наборами ключевых слов. Кроме того, объекты такой системы связаны между собой различными отношениями, например, для научной публикации это может быть список соавторов, для научного работника - список проектов, в выполнении которых он принимал участие, или список конференций, в которых он принимал участие. Публикации, персоналии, научные проекты и конференции в данном примере являются объектами и, следовательно, могут иметь собственные наборы ключевых слов. 

Побудительным мотивом и конечной целью исследований, результаты которых представлены в настоящей диссертации, является создание интеллектуального программного модуля, встраиваемого в наукометрическую информационно-аналитическую систему, способного по имеющимся в системе ключевым словам и определенным связям между ними выявлять семантическую информацию и с ее помощью решать задачи информационного поиска и классификации. Следует также отметить то обстоятельство, что зачастую информационные системы не обладают большим объемом данных для анализа, что делает затруднительным качественное семантическое сравнение объектов. Однако и в таких системах необходимо уметь точно определять релевантную пользователю информацию. Как следствие, важным требованием к разрабатываемому модулю является его способность эффективно работать в условиях ограниченного объема входной информации.

Кроме того, отсутствие достаточного объема данных в реальных системах показывает актуальность и востребованность на практике исследования, результаты которого представлены в настоящей диссертации.

{\aim}  диссертационной работы является исследование и разработка математических моделей, алгоритмов и программных средств интеллектуального анализа наборов ключевых слов, характеризующих объекты в наукометрических интеллектуальных системах, с использованием методов из теории графов и дополнительной информации онтологического характера об объектах в системе. Такая деятельность соответствует областям исследования, отмеченным в пп. 1, 2, 5, 9 Паспорта специальности 05.13.17 – теоретические основы информатики.

{\workrequirements}
Согласно стандарту \mbox{ГОСТ Р ИСО/МЭК 9126-93} к качеству разрабатываемой системы интеллектуального анализа объектов информационной системы предъявляются следующие требования:
\begin{itemize}
    \item широкие функциональные возможности; 
    \item надежность;
    \item эргономичность;
    \item эффективность;
    \item сопровождаемость;
    \item мобильность.
\end{itemize}

Более подробно изложенные выше пункты определены в приложении {\ref{AppendixRequirements}}. Описанные в данном приложении характеристики определяют отличительные стороны решаемой в настоящей диссертации задаче от известных работ, связанных с выделением семантической информации между объектами информационных систем. Существующие системы в большинстве своем опираются на обилие входных данных, к числу которых относятся:
\begin{itemize}
    \item текстовая информация, а именно - аннотации, заголовки, полные тексты документов;
    \item общие объемы системы, которые характеризуются значительным количеством сущностей внутри системы и число связей между ними.
\end{itemize}

В то же время, разрабатываемый программный комплекс является более гибким решением для систем, не обладающим большим объемом данных. Такие системы с одной стороны не содержат в себе огромного количества различных объектов. С другой стороны, о каждом из объектов известно минимальное количество информации - сущности таких систем должны лишь обладать описывающим их набором ключевых слов, либо быть соединены внутренними связями с сущностями, которым набор ключевых слов ассоциирован. Кроме того, разработанные подходы позволяют получать узконаправленные семантические модели для конкретной области знаний. Ручной труд при внедрении таких систем сводится к минимуму.

В разделе {\ref{related_work_concl}} содержится краткое изложение мотивации предъявленных требований к системе, разрабатываемой в данной работе. Кроме того, представлены недостатски существующих методов решения подобных задач. Описываются проблемные места, которые не позволяют применять эти подходы к некоторому классу систем. В конечном итоге выделяется \textbf{специфика} разрабатываемого комплекса, отличающего его от известных аналогов. На основе анализа перечисленных требований была разработана методология решения поставленной задачи.

{\methods} включает следующие характеризующие ее аспекты.
\begin{itemize}
    \item \textbf{Концептуальные положения.} 
        \begin{itemize}
            \item Опора на наборы ключевых слов, ассоциированых с объектами информационной системы.
            \item Возможность использования различных информационных объектов (источников) - НИР, публикации, патенты и т.п..
            \item Наличие механизмов, позволяющих в автоматизированном режиме проводить экспертные оценки адекватности полученных решений.
        \end{itemize}
    \item \textbf{Модели, методы и средства достижения цели.}
        \begin{itemize}
            \item Модели, реализующие концептуальные положения создания и развития на основе графовых представлений данных и эвристических алгоритмов над ними, работающие в условиях отсутствия строгого математического описания.
            \item Методы машинного обучения, необходимые для улучшения качества определения семантических связей между объектами информационной системы;
        \end{itemize}
    \item \textbf{Инструментальные средства.} Для разработки программных комплексов, решающих поставленные задачи, использованы открытые математические, графовые библиотеки, программные пакеты для обработки естественного языка, программные реализации моделей машинного обучения с отрытым исходным кодом.
    %\item \textbf{Архитектурно-технологические решения.}
    \item \textbf{Перечень и постановка задач, решение которых обеспечивает достижение цели.}
        \begin{enumerate}
        \item \hl{Разработка графовой модели представления данных.} Необходимо представить данные системы в виде множества графов, вершинами которых являются некоторые понятия (ключевые слова/наборы слов/сущности системы), а ребрами - отношения между ними. Такие графы необходимы для вычисления различных характеристик для пар понятий. Решение задачи приводится в главе \ref{chapt_word_similarity}.
        \item \hl{Разработка моделей} определения семантической близости пары ключевых слов. Для этого используются построенные графы, разработанные подходы и технологии машинного обучения. \hl{Формируется набор количественных признаков и решающее правило, определяющее степень семантической близости по этому набору.}. Описание разработанных моделей приводится в главе {\ref{chapt_word_similarity}}.
        \item \hl{Разработка моделей} определения семантической близости пары наборов ключевых слов. Разработанные модели используют различные графовые представления, подходы и модели, рассмотренные в предыдущих пунктах. Решению этой задачи посвящена глава \ref{chapt_tuple_similarity}.
        \item \hl{Апробация разработанных моделей.} Используя функцию близости наборов ключевых слов и отношения между сущностями системы, решаются прикладные задачи определения семантической близости пары сущностей. Этой задаче посвящается глава \ref{chapt_applications}.
        \end{enumerate}
\end{itemize}

Для достижения поставленных целей и удовлетворения описанных выше требований были рассмотрены различные методы решения, их преимущества и недостатки. По окончании поиска была составлена методология исследования,  наиболее подходящая поставленным в настоящей диссертации задачам в рамках имеющихся особенностей и ограничений в наборов исходных данных. Подробная мотивация выбранной методологии описывается в \ref{methodology}.

%Первым важным решением было использование графовой схемы представления данных. При этом во внимание принимались следующие аргументы:
%\begin{itemize}
%    \item в связи с тем, что информационная система может не располагать большими объемами данных, важно в полной мере использовать каждый объект и каждую связь в системе;
%    \item для этого подходящим является графовое представление данных, поскольку по графам можно восстановить частичную недостаточность данных. Отсутствие реальной связи между объектами в силу незаполненности системы можно восполнить, проанализировав пути и отношения в графе;
%    \item в следствие небольших объемов входных данных, графовое представление является эффективным способом хранения информации;
%    \item между сущностями возможны разные типы отношений. Построение одного графа для нескольких таких типов может позволить эффективно пользоваться взаимодействием между этими отношениями;
%    \item графовое представление открывает огромные возможности по подсчету различных характеристик для пары сущностей, оцениваемых на уровень похожести. Появляется возможность подсчета пути, количеств различных путей, мер центральностей, соседств, потоков, различных графовых индексов, кластерного анализа  и использования всего богатого инструментария теории графов. Все это может оказать положительный эффект на качество определения семантической близости.
%\end{itemize}

%. То есть сначала решаются базовые задачи определения близости пары слов, затем процесс поднимается на более высокий уровень определения близости наборов, после чего происходит переход к поставленным в диссертации задачам. Мотивировка данного решения излагается далее.


В работе применяются методы анализа текстов на естественном языке, методы машинного обучения и программной инженерии. При изложении результатов диссертационной работы широко используется аппарат теории графов, а также математической логики и математической статистики.

{\novelty}
работы определяется тем, что автором разработаны новые алгоритмы определения семантической близости для пары ключевых слов, а также для пары наборов ключевых слов, описывающих объекты интеллектуальной наукометрической системы. Созданы уникальные методы автоматической генерации обучающей выборки, а также методы автоматической проверки качества работы программных реализаций алгоритмов определения семантически похожих ключевых слов и алгоритмов выявления кластеров близких понятий. Последнее обстоятельство важно, поскольку тестирование программ в данной предметной области требовательно к наличию специалистов, способных точно определить степень близости для пары объектов или понятий. Разработаны алгоритмы построения иерархических классификаторов научных направлений в автоматическом режиме, использующие исключительно наборы ключевых слов. Важными особенностями  указанных алгоритмов являются: отсутствие необходимости больших объемов данных для обучения моделей с приемлемым уровнем качества; возможность использования разработанных моделей для произвольных интеллектуальных систем, использующих ключевые слова для описания сущностей; небольшие человеческие трудозатраты для выставления экспертных оценок. Проведена работа по уменьшению числа параметров системы, что делает разработанные модели и программные средства эргономичными и легкими для настройки.

На основе исследованний о семантической похожести ключевых и дополнительной информации онтологического характера об объектах, подлежащих анализу, решен ряд значимых задачи и востребованных на практике задач.

Доказана вычислительная сложность разработанных алгоритмов, подтверждающая их адекватность (соответствие) требованиям, предъявляемым к разрабатываемому программному комплексу.

{\influence} Рассматриваемый в работе программный комплекс для анализа, обработки и поиска объектов интеллектуальных информационных систем по ключевым словам представляет собой самостоятельный инновационный продукт. Он может использоваться не только в системе, рассматриваемой в данной диссертации, но и в любой информационно-аналитической системе, объекты которой описываются наборами ключевых слов. Кроме того, разработанные методики обработки связей между объектами могут быть перенесены на другие задачи анализа взаимосвязанных объектов. Рассматриваемый программный модуль определения семантической близости между словами  порождает словарь синонимов той области, на которой был обучен. Этот словарь может быть использован в самых разнообразных задачах информационного поиска и обработки естественного языка, и потенциально может привнести дополнительный полезный сигнал для моделей классификации, ранжирования и кластеризации текстовых или текстово-аннотированных объектов.


\textbf{На защиту выносятся} следующие положения:
\begin{itemize}
    \item методы решения задачи определения семантической близости между парой понятий интеллектуальной, использующие ассоциированные с понятиями ключевые слова, а также  и различными отношениями между этими понятиями;
    \item обоснование актуальности решаемой задачи;
    \item методология, принятая для исследования;
    \item научная новизна и практическая значимость диссертации;
    \item результаты исследований, полный перечень которых приведен в заключении.
\end{itemize}

{\probation} Результаты диссертации докладывались на всероссийской конференции с международным участием <<Знания–Онтологии–Теории (ЗОНТ-2016)>>, на международной конференции <<Ломоносовские чтения>> (2014, 2016, 2018),  на механико-математическом факультете МГУ имени М.В. Ломоносова на семинаре «Проблемы современных информационно-вычислительных систем» под руководством д.ф.-м.н., проф. В.А. Васенина (2013, 2015, 2017, 2018), на семинаре \emph{(не нашел название семинара)} в ВЦ РАН под руководством д.ф.-м.н., проф. В.А.Серебрякова (2018).

По теме диссертации опубликовано 7 работ, в том числе одна в зарубежном издании. Четыре статьи \cite{lunev_paper_1,lunev_paper_2,lunev_paper_3,lunev_paper_4} опубликованы в изданиях из перечня ВАК.
%{\contribution} Автор принимал активное участие \ldots

 % Характеристика работы по структуре во введении и в автореферате не отличается (ГОСТ Р 7.0.11, пункты 5.3.1 и 9.2.1), потому её загружаем из одного и того же внешнего файла, предварительно задав форму выделения некоторым параметрам


\textbf{Объем и структура работы.} Диссертация состоит из~введения, четырёх глав, заключения и~двух приложений.
%% на случай ошибок оставляю исходный кусок на месте, закомментированным
%Полный объём диссертации составляет  \ref*{TotPages}~страницу с~\totalfigures{}~рисунками и~\totaltables{}~таблицами. Список литературы содержит \total{citenum}~наименований.
%
Полный объём диссертации составляет
\formbytotal{TotPages}{страниц}{у}{ы}{}, включая
\formbytotal{totalcount@figure}{рисун}{ок}{ка}{ков} и
\formbytotal{totalcount@table}{таблиц}{у}{ы}{}.   Список литературы содержит  
\formbytotal{citenum}{наименован}{ие}{ия}{ий}.
