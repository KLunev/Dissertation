
\chapter*{Введение}							% Заголовок
\addcontentsline{toc}{chapter}{Введение}	% Добавляем его в оглавление
\nocite{*}
Основными задачами современных информационных систем является эффективная организация сбора, хранения, систематизации, поиска и анализа данных.
На настоящее время наиболее представительно и массово востребованные из таких систем способны хранить огромные объемы данных. Стремительный рост хранящейся в них информации приводит к необходимости исследования методов и инструментальных средств разработки программных комплексов, более эффективно решающих задачи организации сбора и хранения, поиска и анализа данных внутри таких больших систем. 

\hl{Исследования, результаты которых представлены в настоящей диссертации, затрагивают важные и востребованные практикой задачи интеллектуального анализа объектов информационно-аналитической наукометрической системы. Автором предлагаются методы решения задачи определения семантической близости объектов, кластеризации объектов, поиска экспертов в различных областях научных знаний, определения тематической направленности объектов. Решение этих задач улучшает качество работы поисковых механизмов, упрощает работу конечного пользователя  с системой, позволяет определять экспертные сообщества и находить коллекции похожих объектов в системе.}

\hl{В рамках проведенных исследований основополагающей является задача определения семантической близости пары объектов. Постановка этой задачи требует определения понятия <<семантически похожих объектов>>. Понятие семантической близости изучалось многими авторами (например, в работах \cite{frawley,sem_DBLP}). Далее под мерой (степенью) смысловой (семантической) близости и похожести (далее - <<близость>>, <<схожесть>>) будет подразумеваться показатель семантического сходства пары рассматриваемых слов или пары наборов слов естественного языка. Здесь следует также отметить, что в контексте рассматриваемой проблемной области мера не всегда является мерой в строгом математическом смысле. В этой сложно формулируемой проблемной области, как правило, активно используются интуитивно понятные эвристические соображения, понятия и основанные на них математические модели и алгоритмы.} 
%%по ключевым словам естественного, ассоциированным с этими объектами,  

Под мерой смысловой схожести в исследовании, результаты которого представлены в настоящей диссертации, будем понимать величину, которая сложно поддается формальному определению. Несмотря на это, интуиция позволяет дать следущее определение паре семантически близких слов: если в речи или в письменном изложении присутствует возможность заменить одно слово на другое так, что смысл предложения не изменится, то два эти слова (заменяемое и замененное) семантически близки. Другими словами, у слушателя возникнет одинаковое представление о цитируемом объекте реального мира в обоих случаях.

Более того, легко дать определение семантически различным словам: после замены одного такого на другое предложение сильно изменяется по смыслу или даже становится абсурдным, то есть теряет какой-либо смысл, даже в том случае, если имеется возможность его <<домыслить>> до некоторого синтаксически корректного предложения \hl{(поставив, например, это слово в подходящую форму).}

Рассмотрим, например, предложение <<пошив мужского костюма>> и последоватьльно будем заменять слово <<костюм>> на слова <<одежда>>, <<фрак>>, <<обувь>>, <<костюмер>>, <<насекомые>>, <<карнавал>>. Если в первых двух случаях заменя кажется разумной: <<пошив мужской одежды>>, <<пошив мужского фрака>>, то третий пример значительно изменяет смысл предложения: <<пошив мужской обуви>>. При замене на четвертое и пятое слова, полученное предложение окончательно теряет смысл.

Исходя из этих соображений, можно всем парам семантически близких слов давать значение меры близости, равное $1$, а всем различным - $0$. Трудность возникает в случаях, когда пара слов не является в рамках данных определений ни парой близких по смыслу, ни парой различных по смыслу слов. Таким парам необходимо ставить некоторое промежуточное значение из интервала $(0, 1)$. В рамках примера, описанного выше, такой парой может являться пара <<костюм-обувь>>. В этот момент возникает неоднозначность в определении того, какое именно значение должна получить данная пара и по какому принципу ранжировать близости различных пар.  Что является более близкими понятиями: пара, связанная отношением гиперонимии (<<стол-мебель>>), пара слов, часто встречающаяся в одних предложениях (<<уголовный-кодекс>>) или <<слова-братья>>, имеющие общего предка-гиперонима (<<декабрь-ноябрь>>)?

Ответ на этот вопрос кроется в постановке задачи, которую мера семантической близости призвана решить. Если рассматривать в качестве системы интернет-магазин и решать внутри этой системе задачу рекомендации товаров, то логично предложить пользователю такой товар, которые часто покупают с тем, что он приобрел. Другими словами, в качестве <<близкого по смыслу>> взять тот, который чаще всего встречается с заданным. Следует однако отметить, что в данном примере предложение товара, абсолютно идентичного по смыслу с уже купленным (тот же самый товар), не имеет смысла.

Другим примером различной трактовки семантической близости может быть классическая поисковая система с программным модулем поисковых расширений. Поисковые расширения - это модуль, добавляющий в текст запроса пользователя новые слова, связанные с запросом. Это позволяет при ранжировании документов вывести на более высокие позиции документы, содержащие эти новые добавленные слова. 

Если эти слова - действительно синонимы  к словам запроса, то скорее всего выдача обогатится и это улучшит ранжирование в целом. Если же расширить слова запроса гиперонимами (например, по запросу <<купить iphone Москва>> для слова <<iphone>> добавить гипероним <<смартфон>>, а для слова <<Москва>> - <<Россия>>, то в выдачу попадут предложения о продаже различных смартфонов (не только iphone), которые, к тому же, будут продаваться не только в Москве, а по всей России. Тем не менее, в некоторых случаях, например, когда задано слово, по которому нет практически никаких документов в базе, добавление документов с гиперонимом к этому слову обычно является разумной идеей.

Однако, самый плохой из возможных сценариев - расширять словами-братьями. В этом случае iphone заменится на (в некотором смысле) близкое по смыслу samsung и пользователь будет весьма опечален: если в прошлой ситуации он попадал на сайты со \emph{смартфонами}, среди которых мог быть нужный ему, то теперь ему целенаправленно показывают на выдаче нерелевантный для него товар.

Кроме того, можно заметить, что уровень близости зависит от тематики той системы, в которой слова употребляются. В системе самой общей тематики слова <<школа-университет>> должны иметь достаточно высокий уровень близости. Если же рассматривать некий образовательный портал, тематическая направленность которого узкоспециальна и относится к образованию, близость данной пары должна быть заметно ниже, потому что в данном контексте это два совершенно разных учебных заведения и различия в данной ситуации имееют принципиальное значение.

Помимо этого, слова могут быть многозначными и даже тривиальная пара <<орган-орган>> может иметь уровень схожести близкий к нулю, если считать, что первое слово - это музыкальный инструмент, а второе - часть тела или термин из юриспруденции (при этом совершенно не очевидно, какое из этих двух значений ближе по смыслу к музыкальному органу).

Описанные выше примеры показывают все неоднозначность трактовки семантической близости на примере слов естественного языка. В рамках данной работы принимается следующее правило: чем слабее изменяется смысл при замене одного слова вторым в различных предложениях, содержащих первое слово, тем больше семантическая близость этой пары слов. 

В этой связи еще раз подчеркивается, что для достижения поставленых в данной диссертации целей, были использованы модели, которые во многом опираются на эвристические, интуитивные понятия. Следует также отметить, что задача разрешения смысловой многозначности в рамках данной работы не рассматривается.

Высоким уровнем близости должны обладать синонимы в привычном значении из языкознания, правильные расшифровки аббревиатур, переводы слова на другие языки, формы одного слова, различные способы написания. В следующей далее таблице \ref{tbl:sim_example} приведены примеры семантически похожих пар ключевых слов из наукометрической системы:
\begin{tabularx}{16cm}{|X|X|X|}
        \hline
        Первое слово & Второе слово & Комментарий \\ \hline
        умение & навык & Синонимия \\ \hline
        полином & многочлен & Синонимия \\ \hline
        β-адреноблокаторы & бета-адреноблокаторы & Различные способы написания одного слова \\ \hline
        орви & острые респираторные вирусные инфекции & Расшифровка аббревиатуры \\ \hline
        хехцир & khekhtsyr & Транслитерация \\ \hline
        корень & корни & Разные формы одного слова \\ \hline
        crisis & кризис & Перевод на другой язык \\ \hline
        cu & медь & Другая форма названия \\ \hline
\caption{Примеры семантически близких ключевых слов} \label{tbl:tuple_test}
        \label{tbl:sim_example}
\end{tabularx}

Одним из известных и широкоиспользуемых способов высокоуровневого описания данных, представленных в системе, является использование \emph{ключевых слов}. Ключевые слова (или теги) - это набор слов естественного языка или терминов, которые коротко описывают отдельный документ, который хранится в информационной системе. Они используются в качестве метаинформации для публикаций (в том числе и научных) в средствах массовой информации и печатных изданий. Такой подход позволяет читателю быстро понять основное направления изложения и концептуальные положения представленной информации, отмечают некоторые понятия и сущности, с помощью которых решаются представленные в этих публикациях задачи. 

Многие современные информационно-коммуникационные структуры, такие как социальные сети, блоговые и поисковые системы, используют ключевые слова для описания содержащихся в них сущностей (объектов). Такой подход значительно упрощает для пользователя поиск необходимых ему объектов системы, потому что позволяет сделать это с помощью запроса к системе на естественном языке. Кроме того, ключевые слова помогают поисковым системам по данному запросу выделять наиболее релевантные объекты системы. К числу таких объектов относятся, например, текстовые документы, изображения, видеозаписи и любой другой объект, которому был приписан набор ключевых слов. Многие исследователи активно занимались и продолжают заниматься анализом ключевых слов в целях кластеризации, визуализации, классификации, индексации и поиска целевых объектов.

Кроме того, ключевые слова можно рассматривать и как классификаторы контента, формирующие тезаурус предметной области, на основе которого этот контент описывается. Примером такого классификатора является \emph{универсальная десятичная классификация (УДК)}, используемая для систематизации и группировки накопленных человечеством знаний по тематическим разделам. Данная классификация различным областям науки, литературы и искусства ставит в соответсвии цифровые коды. Описание областей задается с помощью небольшого набора ключевых слов, характеризующих данное направление. Данные УДК построены по иерархическому принципу: более общие направления науки (а также соответствующием им коды) описываются общими по смыслу словами, например, <<Общественные науки>>. При углублении и выборе определенной специализации внутри данного направления, описание приводится с помощью более конкретных понятий, таких как <<Политика>>, <<Право>>, <<Экономика>>, <<Народное хозяйство>> и т.д.. Таким образом, с помощью небольшого множества ключевых слов появляется возможность структуризации необходимого любого направления и соответсвующего ему кода. Несмотря на все многообразие тематик, поиск необходимого кода не занимает много времени, что является возможным благодаря использованию ключевых слов.

\hl{Важным является то обстоятельство, что реальные информационно-аналитические системы во многих случаях не обладают достаточным объемом данных для анализа. Рассмотрим, например, научные публикации, как объекты наукометрической системы. Зачастую в таких данных отсутствует полнотекстовая информация. Доступной в этом случае является лишь метаинформация: авторы, название, ключевые слова. В связи с этим, в рамках проводимых исследований информация об объекте ограничивается набором ключевых слов на естественном языке, а также связями данного объекта с другими объектами системы. Другими словами, каждому объекту системы ставится в соответствии набор ключевых слов. Семантическая близость между объектами такой системы сводится к семантической близости между соответствующими им наборами ключевых слов. В свою очередь, семантическая близость между парой наборов ключевых слов опирается на семантическую близость между словами, входящими в эти наборы.} 

\hl{Кроме того, отмечается, что каждый объект описывается очень малым объемом данных (обычно это 5-6 слов). Это обстоятельство вносит существенные ограничения в допустимые методы решения обозначенных выше задач.}




\newcommand{\actuality}{}
\newcommand{\progress}{}
\newcommand{\aim}{{\textbf\aimTXT}}
\newcommand{\tasks}{\textbf{\tasksTXT} }
\newcommand{\novelty}{\textbf{\noveltyTXT}}
\newcommand{\influence}{\textbf{\influenceTXT}}
\newcommand{\methods}{\textbf{\methodsTXT}}
\newcommand{\defpositions}{\textbf{\defpositionsTXT}}
\newcommand{\reliability}{\textbf{\reliabilityTXT}}
\newcommand{\probation}{\textbf{\probationTXT}}
\newcommand{\contribution}{\textbf{\contributionTXT}}
\newcommand{\publications}{\textbf{\publicationsTXT}}
\newcommand{\workrequirements}{\textbf{\workrequirementsTXT}}


{\actuality} Обзор, введение в тему

{\aim} данной работы является \ldots


{\novelty}
\begin{enumerate}
  \item Впервые \ldots
  \item Впервые \ldots
  \item Было выполнено оригинальное исследование \ldots
\end{enumerate}

{\influence} \ldots

{\methods} \ldots

{\probation} \ldots

{\contribution} Автор принимал активное участие \ldots

 % Характеристика работы по структуре во введении и в автореферате не отличается (ГОСТ Р 7.0.11, пункты 5.3.1 и 9.2.1), потому её загружаем из одного и того же внешнего файла, предварительно задав форму выделения некоторым параметрам


\textbf{Объем и структура работы.} Диссертация состоит из~введения, четырёх глав, заключения и~двух приложений.
%% на случай ошибок оставляю исходный кусок на месте, закомментированным
%Полный объём диссертации составляет  \ref*{TotPages}~страницу с~\totalfigures{}~рисунками и~\totaltables{}~таблицами. Список литературы содержит \total{citenum}~наименований.
%
Полный объём диссертации составляет
\formbytotal{TotPages}{страниц}{у}{ы}{}, включая
\formbytotal{totalcount@figure}{рисун}{ок}{ка}{ков} и
\formbytotal{totalcount@table}{таблиц}{у}{ы}{}.   Список литературы содержит  
\formbytotal{citenum}{наименован}{ие}{ия}{ий}.
