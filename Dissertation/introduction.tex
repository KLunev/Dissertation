\chapter*{Введение}							% Заголовок
\addcontentsline{toc}{chapter}{Введение}	% Добавляем его в оглавление
\nocite{*}
%Современные информационные системы позво
Основными задачами современных информационных систем является эффективная организация хранения, поиска и анализа информации.
На сегодняшний день наибольшие из таких систем способны хранить в себе данныe огромного размера. Стремительный рост объемов информации ведет к необходимости исследования методов и разработки программных комплексов, более эффективно решающих задачи хранения, анализа и поиска данных внутри системы. 
Ключевые слова (или теги) - это набор слов естественного языка или терминов, которые коротко описывают документ информационной системы. Они используются в качестве метаинформации для публикаций (в том числе и научных), что позволяет читателю быстро понять основное направления изложения и концепцию представленной информации, а также указывают некоторые понятия, с помощью которых решаются поставленные задачи. Кроме этого, ключевые слова можно рассматривать как классификаторы контента.

Многие современные информационно-коммуникационные структуры, такие как социальные сети, блоговые и поисковые системы, используют ключевые слова для описания содержащихся в них сущностей (объектов). Такой подход значительно упрощает для пользователя поиск необходимых ему объектов системы, потому что позволяет сделать это с помощью запроса к системе на естественном языке.  Кроме того, ключевые слова помогают поисковым системам по данному запросу выделять наиболее релевантные объекты системы. К числу таких объектов относятся, например, текстовые документы, изображения, видеозаписи и любой другой объект, которому был приписан набор ключевых слов. Многие исследователи активно занимались и продолжают заниматься анализом ключевых слов в целях кластеризации, визуализации, классификации, индексации и поиска целевых объектов.

Исследования, результаты которых представлены в настоящей диссертации, затрагивают важную и востребованную практикой задачу кластеризации объектов по ключевым словам, ассоциированным с этими объектами. Её решение помогает находить в больших информационных коллекциях кластера похожих объектов, удалять дубликаты документов, определять экспертные сообщества. Далее под мерой (степенью) смысловой близости и похожести (далее - <<близость>>, <<схожесть>>) будет подразумеваться показатель семантического сходства пары рассматриваемых ключевых слов или набора слов естественного языка.




\newcommand{\actuality}{}
\newcommand{\progress}{}
\newcommand{\aim}{{\textbf\aimTXT}}
\newcommand{\tasks}{\textbf{\tasksTXT} }
\newcommand{\novelty}{\textbf{\noveltyTXT}}
\newcommand{\influence}{\textbf{\influenceTXT}}
\newcommand{\methods}{\textbf{\methodsTXT}}
\newcommand{\defpositions}{\textbf{\defpositionsTXT}}
\newcommand{\reliability}{\textbf{\reliabilityTXT}}
\newcommand{\probation}{\textbf{\probationTXT}}
\newcommand{\contribution}{\textbf{\contributionTXT}}
\newcommand{\publications}{\textbf{\publicationsTXT}}


{\actuality} В качестве предмета исследования и анализа в диссертации выступают объекты наукометрической информационно-аналитической системы, которые описываются наборами ключевых слов. Кроме того, объекты такой системы связаны между собой различными отношениями, например, для научной публикации это может быть список соавторов, для научного работника - список проектов, в выполнении которых он принимал участие, или список конференций, в которых он принимал участие. Публикации, персоналии, научные проекты и конференции в данном примере являются объектами и, следовательно, могут иметь собственные наборы ключевых слов. 

Побудительным мотивом и конечной целью исследований, результаты которых представлены в настоящей диссертации, является создание интеллектуального программного модуля, встраиваемого в наукометрическую информационно-аналитическую систему, способного по имеющимся в системе ключевым словам и определенным связям между ними выявлять семантическую информацию и с ее помощью решать задачи информационного поиска и классификации. Следует также отметить то обстоятельство, что зачастую информационные системы не обладают большим объемом данных для анализа, что делает затруднительным качественное семантическое сравнение объектов. Однако и в таких системах необходимо уметь точно определять релевантную пользователю информацию. Как следствие, важным требованием к разрабатываемому модулю является его способность эффективно работать в условиях ограниченного объема входной информации.

Кроме того, отсутствие достаточного объема данных в реальных системах показывает актуальность и востребованность на практике исследования, результаты которого представлены в настоящей диссертации.

{\aim}  диссертационной работы является исследование и разработка математических моделей, алгоритмов и программных средств интеллектуального анализа наборов ключевых слов, характеризующих объекты в наукометрических интеллектуальных системах, с использованием методов из теории графов и дополнительной информации онтологического характера об объектах в системе. Такая деятельность соответствует областям исследования, отмеченным в пп. 1, 2, 5, 9 Паспорта специальности 05.13.17 – теоретические основы информатики.

{\workrequirements}
Согласно стандарту \mbox{ГОСТ Р ИСО/МЭК 9126-93} к качеству разрабатываемой системы интеллектуального анализа объектов информационной системы предъявляются следующие требования:
\begin{itemize}
    \item широкие функциональные возможности; 
    \item надежность;
    \item эргономичность;
    \item эффективность;
    \item сопровождаемость;
    \item мобильность.
\end{itemize}

Более подробно изложенные выше пункты определены в приложении {\ref{AppendixRequirements}}. Описанные в данном приложении характеристики определяют отличительные стороны решаемой в настоящей диссертации задаче от известных работ, связанных с выделением семантической информации между объектами информационных систем. Существующие системы в большинстве своем опираются на обилие входных данных, к числу которых относятся:
\begin{itemize}
    \item текстовая информация, а именно - аннотации, заголовки, полные тексты документов;
    \item общие объемы системы, которые характеризуются значительным количеством сущностей внутри системы и число связей между ними.
\end{itemize}

В то же время, разрабатываемый программный комплекс является более гибким решением для систем, не обладающим большим объемом данных. Такие системы с одной стороны не содержат в себе огромного количества различных объектов. С другой стороны, о каждом из объектов известно минимальное количество информации - сущности таких систем должны лишь обладать описывающим их набором ключевых слов, либо быть соединены внутренними связями с сущностями, которым набор ключевых слов ассоциирован. Кроме того, разработанные подходы позволяют получать узконаправленные семантические модели для конкретной области знаний. Ручной труд при внедрении таких систем сводится к минимуму.

В разделе {\ref{related_work_concl}} содержится краткое изложение мотивации предъявленных требований к системе, разрабатываемой в данной работе. Кроме того, представлены недостатски существующих методов решения подобных задач. Описываются проблемные места, которые не позволяют применять эти подходы к некоторому классу систем. В конечном итоге выделяется \textbf{специфика} разрабатываемого комплекса, отличающего его от известных аналогов. На основе анализа перечисленных требований была разработана методология решения поставленной задачи.

{\methods} включает следующие характеризующие ее аспекты.
\begin{itemize}
    \item \textbf{Концептуальные положения.} 
        \begin{itemize}
            \item Опора на наборы ключевых слов, ассоциированых с объектами информационной системы.
            \item Возможность использования различных информационных объектов (источников) - НИР, публикации, патенты и т.п..
            \item Наличие механизмов, позволяющих в автоматизированном режиме проводить экспертные оценки адекватности полученных решений.
        \end{itemize}
    \item \textbf{Модели, методы и средства достижения цели.}
        \begin{itemize}
            \item Модели, реализующие концептуальные положения создания и развития на основе графовых представлений данных и эвристических алгоритмов над ними, работающие в условиях отсутствия строгого математического описания.
            \item Методы машинного обучения, необходимые для улучшения качества определения семантических связей между объектами информационной системы;
        \end{itemize}
    \item \textbf{Инструментальные средства.} Для разработки программных комплексов, решающих поставленные задачи, использованы открытые математические, графовые библиотеки, программные пакеты для обработки естественного языка, программные реализации моделей машинного обучения с отрытым исходным кодом.
    %\item \textbf{Архитектурно-технологические решения.}
    \item \textbf{Перечень и постановка задач, решение которых обеспечивает достижение цели.}
        \begin{enumerate}
        \item \hl{Разработка графовой модели представления данных.} Необходимо представить данные системы в виде множества графов, вершинами которых являются некоторые понятия (ключевые слова/наборы слов/сущности системы), а ребрами - отношения между ними. Такие графы необходимы для вычисления различных характеристик для пар понятий. Решение задачи приводится в главе \ref{chapt_word_similarity}.
        \item \hl{Разработка моделей} определения семантической близости пары ключевых слов. Для этого используются построенные графы, разработанные подходы и технологии машинного обучения. \hl{Формируется набор количественных признаков и решающее правило, определяющее степень семантической близости по этому набору.}. Описание разработанных моделей приводится в главе {\ref{chapt_word_similarity}}.
        \item \hl{Разработка моделей} определения семантической близости пары наборов ключевых слов. Разработанные модели используют различные графовые представления, подходы и модели, рассмотренные в предыдущих пунктах. Решению этой задачи посвящена глава \ref{chapt_tuple_similarity}.
        \item \hl{Апробация разработанных моделей.} Используя функцию близости наборов ключевых слов и отношения между сущностями системы, решаются прикладные задачи определения семантической близости пары сущностей. Этой задаче посвящается глава \ref{chapt_applications}.
        \end{enumerate}
\end{itemize}

Для достижения поставленных целей и удовлетворения описанных выше требований были рассмотрены различные методы решения, их преимущества и недостатки. По окончании поиска была составлена методология исследования,  наиболее подходящая поставленным в настоящей диссертации задачам в рамках имеющихся особенностей и ограничений в наборов исходных данных. Подробная мотивация выбранной методологии описывается в \ref{methodology}.

%Первым важным решением было использование графовой схемы представления данных. При этом во внимание принимались следующие аргументы:
%\begin{itemize}
%    \item в связи с тем, что информационная система может не располагать большими объемами данных, важно в полной мере использовать каждый объект и каждую связь в системе;
%    \item для этого подходящим является графовое представление данных, поскольку по графам можно восстановить частичную недостаточность данных. Отсутствие реальной связи между объектами в силу незаполненности системы можно восполнить, проанализировав пути и отношения в графе;
%    \item в следствие небольших объемов входных данных, графовое представление является эффективным способом хранения информации;
%    \item между сущностями возможны разные типы отношений. Построение одного графа для нескольких таких типов может позволить эффективно пользоваться взаимодействием между этими отношениями;
%    \item графовое представление открывает огромные возможности по подсчету различных характеристик для пары сущностей, оцениваемых на уровень похожести. Появляется возможность подсчета пути, количеств различных путей, мер центральностей, соседств, потоков, различных графовых индексов, кластерного анализа  и использования всего богатого инструментария теории графов. Все это может оказать положительный эффект на качество определения семантической близости.
%\end{itemize}

%. То есть сначала решаются базовые задачи определения близости пары слов, затем процесс поднимается на более высокий уровень определения близости наборов, после чего происходит переход к поставленным в диссертации задачам. Мотивировка данного решения излагается далее.


В работе применяются методы анализа текстов на естественном языке, методы машинного обучения и программной инженерии. При изложении результатов диссертационной работы широко используется аппарат теории графов, а также математической логики и математической статистики.

{\novelty}
работы определяется тем, что автором разработаны новые алгоритмы определения семантической близости для пары ключевых слов, а также для пары наборов ключевых слов, описывающих объекты интеллектуальной наукометрической системы. Созданы уникальные методы автоматической генерации обучающей выборки, а также методы автоматической проверки качества работы программных реализаций алгоритмов определения семантически похожих ключевых слов и алгоритмов выявления кластеров близких понятий. Последнее обстоятельство важно, поскольку тестирование программ в данной предметной области требовательно к наличию специалистов, способных точно определить степень близости для пары объектов или понятий. Разработаны алгоритмы построения иерархических классификаторов научных направлений в автоматическом режиме, использующие исключительно наборы ключевых слов. Важными особенностями  указанных алгоритмов являются: отсутствие необходимости больших объемов данных для обучения моделей с приемлемым уровнем качества; возможность использования разработанных моделей для произвольных интеллектуальных систем, использующих ключевые слова для описания сущностей; небольшие человеческие трудозатраты для выставления экспертных оценок. Проведена работа по уменьшению числа параметров системы, что делает разработанные модели и программные средства эргономичными и легкими для настройки.

На основе исследованний о семантической похожести ключевых и дополнительной информации онтологического характера об объектах, подлежащих анализу, решен ряд значимых задачи и востребованных на практике задач.

Доказана вычислительная сложность разработанных алгоритмов, подтверждающая их адекватность (соответствие) требованиям, предъявляемым к разрабатываемому программному комплексу.

{\influence} Рассматриваемый в работе программный комплекс для анализа, обработки и поиска объектов интеллектуальных информационных систем по ключевым словам представляет собой самостоятельный инновационный продукт. Он может использоваться не только в системе, рассматриваемой в данной диссертации, но и в любой информационно-аналитической системе, объекты которой описываются наборами ключевых слов. Кроме того, разработанные методики обработки связей между объектами могут быть перенесены на другие задачи анализа взаимосвязанных объектов. Рассматриваемый программный модуль определения семантической близости между словами  порождает словарь синонимов той области, на которой был обучен. Этот словарь может быть использован в самых разнообразных задачах информационного поиска и обработки естественного языка, и потенциально может привнести дополнительный полезный сигнал для моделей классификации, ранжирования и кластеризации текстовых или текстово-аннотированных объектов.


\textbf{На защиту выносятся} следующие положения:
\begin{itemize}
    \item методы решения задачи определения семантической близости между парой понятий интеллектуальной, использующие ассоциированные с понятиями ключевые слова, а также  и различными отношениями между этими понятиями;
    \item обоснование актуальности решаемой задачи;
    \item методология, принятая для исследования;
    \item научная новизна и практическая значимость диссертации;
    \item результаты исследований, полный перечень которых приведен в заключении.
\end{itemize}

{\probation} Результаты диссертации докладывались на всероссийской конференции с международным участием <<Знания–Онтологии–Теории (ЗОНТ-2016)>>, на международной конференции <<Ломоносовские чтения>> (2014, 2016, 2018),  на механико-математическом факультете МГУ имени М.В. Ломоносова на семинаре «Проблемы современных информационно-вычислительных систем» под руководством д.ф.-м.н., проф. В.А. Васенина (2013, 2015, 2017, 2018), на семинаре \emph{(не нашел название семинара)} в ВЦ РАН под руководством д.ф.-м.н., проф. В.А.Серебрякова (2018).

По теме диссертации опубликовано 7 работ, в том числе одна в зарубежном издании. Четыре статьи \cite{lunev_paper_1,lunev_paper_2,lunev_paper_3,lunev_paper_4} опубликованы в изданиях из перечня ВАК.
%{\contribution} Автор принимал активное участие \ldots

 % Характеристика работы по структуре во введении и в автореферате не отличается (ГОСТ Р 7.0.11, пункты 5.3.1 и 9.2.1), потому её загружаем из одного и того же внешнего файла, предварительно задав форму выделения некоторым параметрам


\textbf{Объем и структура работы.} Диссертация состоит из~введения, четырёх глав, заключения и~двух приложений.
%% на случай ошибок оставляю исходный кусок на месте, закомментированным
%Полный объём диссертации составляет  \ref*{TotPages}~страницу с~\totalfigures{}~рисунками и~\totaltables{}~таблицами. Список литературы содержит \total{citenum}~наименований.
%
Полный объём диссертации составляет
\formbytotal{TotPages}{страниц}{у}{ы}{}, включая
\formbytotal{totalcount@figure}{рисун}{ок}{ка}{ков} и
\formbytotal{totalcount@table}{таблиц}{у}{ы}{}.   Список литературы содержит  
\formbytotal{citenum}{наименован}{ие}{ия}{ий}.
