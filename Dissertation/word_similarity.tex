\chapter{Определение смысловой близости пары ключевых слов} \label{chapt1}
В настоящем разделе подробно описываются алгоритмы определения семантической близости пары ключевых слов.
Для определения близости вводятся вспомогательные графы, вершинами которых являются ключевые слова, а ребра указывают на некоторые свойства пары ключевых слов.
На основе построенных графов с помощью методов из теории графов вычисляются различные количественные характеристики, необходимые для выявления семантической связи между рассматриваемыми словами.
После этого предлагаются способы применения техник машинного обучения, которые в значительной мере улучшают качество определения смысловой близости между ключевыми словами. 
Также описывается новый алгоритм автоматического формирования обучающей выборки для машинного обучения. Важность данного алгоритма в том, что он избавляет от необходимости ручной разметки данных, которая обычно является трудозатратной работой.
В конце раздела представлены результаты тестовых испытаний программных реализаций алгоритмов, выводы о выполненной работе, а также предлагаются идеи для дальнейшего улучшения качества определения семантической близости наборов ключевых слов.

\section{Построение графовой модели близости пары ключевых слов}
\subsection{Построение графа близости ключевых слов} %\label{sect1_1}
\subsection{Построение контекстного графа ключевых слов} %\label{sect1_1}
\subsection{Графовые методы выявление тематических направлений в наборах ключевых слов} %\label{sect1_2}
\subsection{Методы кластеризации ключевых слов по графам ключевых слов}
\subsection{Тестовые испытания}
\section{Использование техник машинного обучения для улучшения модели близости слов}
\subsection{Методы формирования обучающей выборки}
\subsubsection{Ручные методы}
\subsubsection{Автоматические методы}
\subsection{Признаковое описание модели машинного обучения}
\subsection{Описание и настройка модели машинного обучения}
\subsection{Тестовые испытания}
\section{Построение тезауруса ключевых слов по коллекции наборов}
\subsection{Алгоритм построения}
\subsection{Тестовые испытания}
\section{Выводы}
