\chapter{Особенности программной реализации модуля анализа ключевых слов в системе}
Настоящее приложение содержит характеристики и показатели, определяющие требования, которые предъявляются к качеству разрабатываемого программного комплекса в соответствии со стандартом \mbox{ГОСТ Р ИСО/МЭК 9126-93}.
\section{Функциональные требования}
Система должна поддерживать следующие функциональные требования.
\begin{enumerate}
    \item  Подготовка моделей определения семантической близости пар объектов системы по имеющимся данным:
    \begin{enumerate}
        \item  модель семантической близости пары ключевых слов;
        \item  модель семантической близости наборов ключевых слов;
        \item  модель семантической близости пары сущностей системы.
    \end{enumerate}
    Самым вычислительно сложным является построение модели близости пары ключевых слов. Каждая следующая модель обучается последовательно, поскольку в значительной степени опирается на предыдущую. Весь этап обучения является алгоритмически сложной задачей и включает в себя несколько этапов:
        \begin{itemize}
            \item предподготовка данных;
            \item построение необходимых графов из данных;
            \item сбор обучающей выборки для моделей;
            \item подсчет многочисленных графовых характеристик по графам для объектов обучающей выборки;
            \item непосредственно обучение модели.
        \end{itemize}
    \item Быстрое обновление имеющихся и добавление новых данных в систему:
    \begin{enumerate}
        \item добавление/модификация наборов ключевых слов;
        \item добавление/модификация дополнительных графов, связывающих сущности системы различными отношениями.
    \end{enumerate}
    При изменении данных системы возникает необходимость переобучения моделей близости для поддержания консистентного состояния между данными и моделями. Сложность данного пункта в том, что наивная переподготовка моделей после изменения данных может занимать продолжительное время. По этой причине возникает необходимость разработки сложных алгоритмов инкрементальной подготовки моделей, что сократит время их дообучения.
    \item Быстрая кластеризация ключевых слов системы и поиск необходимого кластера. Данный процесс происходит после обучения модели близости пары ключевых слов.
    \item Быстрый поиск похожих объектов с помощью обученных моделей:
    \begin{enumerate}
        \item поиск наиболее похожих ключевых слов к заданному;
        \item поиск сущностей, релевантных заданному набору ключевых слов;
        \item поиск набора ключевых слов, подходящих для заданной сущности.
    \end{enumerate}
    \item Реализация подмодулей, решающих практически значимые задачи информационного поиска в рамках аналитической системы.
    \begin{enumerate}
        \item Подмодуль поиска эксперта. Реализация функционала поиска сущностей информационной системы, релевантных поисковому запросу из ключевых слов.
        \item Подмодуль предложенных ключевых слов. Реализация функционала предложения пользователю новых ключевых слов по словам, введенным на данный момент или по имеющейся связанной информации.
    \end{enumerate}
    \item Сбор пользовательской информации в ходе взаимодействия с комплексом.
    \begin{enumerate}
        \item Подмодуль поиска эксперта. Логирование релевантных и нерелевантных по мнению пользователя результатов.
        \item Подмодуль предложенных ключевых слов. Логирование выбранных и невыбранных пользователем ключевых слов из числа предложенных.
    \end{enumerate}
    Поиск должен выполняться как только пользователь ввел запрос и подтвердил его. Вычисление и показ результатов должны укладываться в несколько секунд. Сложность данного требования в том, что для каждого запроса необходимо подсчитать огромное число графовых характеристик и применить соответствующую предобученную графовую модель близости. В следствии этого данный пункт представляет собой сложную техническую задачу по оптимизации вычислений.
    \item Соответствие принятым в индустрии соглашениям и стандартам.
    \item Высокий уровень доверия и защищенность ресурсов системы от деструктивных воздействий путем соблюдения положений (стандартов) информационной и функциональной безопасности, а также использования программного обеспечения с открытым исходным кодом.
    \item Система должна функционировать под управлением ОС с открытым исходным кодом.
\end{enumerate}

\section{Надежность}
Следующие свойства должны быть удовлетворены.
\begin{enumerate}
    \item  Качество обученных моделей должно валидироваться на отложенных выборках после каждого изменения моделей.
        \begin{enumerate}
            \item Для каждой модели и соответствующих ей наборов тестов определяется необходимый уровень качества по выбранным метрикам и уровень производительности и величину ресурсозатратности.
            \item Для каждой модели выбирается отложенное множество объектов, на которых модель применяется. При обновлении данных, переобученные модели применяются к тому же множеству объектов и автоматически проверяется, что изменения в предсказаниях оказываются ниже определенного порога. Если это условие не выполняется, то эксперту по системе необходимо детально разбираться в причинах сильных отклонений в предсказаниях. Таким образом в системе реализуется регрессионное тестирование.
        \end{enumerate}
    \item  Стабильная работа в условиях одновременного использования сотрудниками крупной организации.
    \item  Устойчивость к программным ошибкам и ошибкам интерфейса.
\end{enumerate}
\section{Практичность}
В отношении разрабатываемого комплекса должно выполняться следующее.
\begin{enumerate}
    \item Комплекс должeн иметь простой интуитивный интерфейс для пользователя.
    \item Комплекс должeн быть легко читаемой и понимаемой для разработчиков.
    \item Комплекс должен включать средства обратной связи пользователя с разработчиками.
\end{enumerate}
\section{Эффективность}
Программный комплекс должен быть эффективен в следующих показателях.
\begin{enumerate}
    \item Удовлетворительные показатели качества работы моделей на сильно ограниченных по объему данных;
    \item Этап предподготовки комплекса:
        \begin{enumerate}
            \item В течение одних суток:
            \begin{enumerate}
                \item пересчет аналитических моделей определения близости ключевых слов, включая подготовку всех необходимых данных;
                \item пересбор тезауруса ключевых слов.
            \end{enumerate}
            \item В течение нескольких часов:
            \begin{enumerate}
                \item обогащение наборов ключевых слов новой информацией;
                \item пересчет аналитических моделей определения близости объектов информационной системы;
                \item быстрое добавления новых отношений между сущностями системы.
            \end{enumerate}
        \end{enumerate}
    \item  Этап использования моделей:
        \begin{enumerate}
            \item  быстрое построение выдачи по пользовательскому запросу;
            \item  быстрое получение кластера ключевых слов содержащее данное;
            \item  быстрая реализация поисковых подсказок при вводе запроса.
        \end{enumerate}
\end{enumerate}
\section{Сопровождаемость}
Выдвигаются следующие требования к разрабатываемому комплексу по сопровождаемости.
\begin{enumerate}
    \item  Весь комплекс архитектурно должен разбиваться на ряд отдельных модулей. Логика и параметры этих модулей системы должны быть инкапсулированы друг от друга.
    \item  Иметь возможность быстрого и эффективного способа расширения функционала комплекса.
    \item  Иметь возможность обновлять входные данные в автоматическом режиме.
    \item  Быть документированной.
\end{enumerate}
\section{Мобильность}
Следующие свойства должны выполняться для разрабатываемого комплекса.
\begin{enumerate}
    \item Возможность внедрения в различные информационно-аналитические системы произвольной направленности с допустимым уровнем качества моделей. Модели должны иметь возможность обучаться на данных новой системы.
    \item Возможность обучения специфических моделей семантической близости, автоматически подстраиваемых к предметной области системы, в которой разворачивается комплекс.
    \item Возможность обучения моделей семантической близости без имеющихся обучающих примеров.
    \item Возможность внедрения в систему с дефицитом данных о ключевых словах.
    \item Адаптируемость к добавлению новых сущностей и отношений между ними в системе.
    \item Развертываемость комплекса внутри новой системы не должна занимать много времени работы экспертов. Необходимо лишь наладить поставку данных в нужном формате и сконфигурать модули для наиболее эффективного решения задач конкретной системы.
    \item Устойчивость к пропускам и неточностям в данных.
\end{enumerate}
